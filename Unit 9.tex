\documentclass[12pt, a4paper]{article}
\usepackage{titlesec}
\usepackage{lipsum}
\usepackage{amssymb}
\usepackage{amsmath}
\usepackage{bbm}
\usepackage[margin=1.3in]{geometry}
\usepackage[mathscr]{euscript}
\usepackage{tabto}
\usepackage{cancel}


\DeclareSymbolFont{rsfs}{U}{rsfs}{m}{n}
\DeclareSymbolFontAlphabet{\mathscrsfs}{rsfs}
\titlelabel{\thetitle.\quad}
\righthyphenmin=1000
\lefthyphenmin=1000


\begin{document}
\section*{Unit 9}
\vspace{1em}

\subsection*{LOS 1. Solve wave equation using different separation techniques}
\subsection*{LOS 2. Understand theorem on an orthonormal basis of eigenfunctions}
Problem:
\begin{align*}
    &u_{tt} = \Delta u\\
    &\left.u\right\rvert_{\partial D} = \phi\\
    &\left.u_t\right\rvert_{\partial D} = \psi\\
\end{align*}
Separation 1:
\begin{align*}
    &u = T(t)V(x)\\
    &T''V = T\Delta V\\
    &T'' = -\lambda T &&\rightarrow T(t) = Ce^{-\lambda t}\\
    &\Delta V = -\lambda V &&(1)\\
\end{align*}
Separation 2:
\begin{align*}
    \text{From (1) } V(r, \theta, \varphi) &= R(r)P(\theta)Q(\varphi)\\
    &= R(r)Y(\theta, \varphi)
\end{align*}
\begin{align*}
    &\Delta V = R_{rr}Y + \frac{2}{r}R_rY + \frac{1}{r^2}\Delta_{\theta, \varphi}Y = -\lambda RY\\
    &r^2\frac{R_{rr}}{R} + 2r\frac{R_r}{R} + r^2\lambda = \frac{\Delta Y}{Y} = -\gamma\\\\
    &r^2\frac{R_{rr}}{R} + 2r\frac{R_r}{R} + r^2\lambda= -\gamma\\
    &R_{rr} + \frac{2}{r}R_r + \left(\lambda - \frac{\gamma}{r^2}\right)R= 0&&(2)\\\\
    &\Delta Y = -\gamma Y &&(3)\\
\end{align*}
Consider $\gamma =0$:
\begin{align*}
    &R_{rr} + \frac{2}{r}R_r + \lambda R= 0\\
    \text{Change of variable: }&w = \sqrt{r}R(r)\\
    &w_r = \sqrt{r}R_r + \frac{1}{2\sqrt{r}}R\\
    &w_{rr} = \sqrt{r}R_{rr} + \frac{1}{2\sqrt{r}}R_r+ \frac{1}{2\sqrt{r}}R_r-\frac{1}{4}r^{-\frac{3}{2}}R\\
\end{align*}
Substitute to (2) to get the Bessel function:
\begin{align*}
    &w_{rr} + \frac{1}{r}w_r + \left(\lambda - \frac{\gamma + \frac{1}{2}}{r^2}\right)w = 0\\
    &\text{Change of variable: $\rho = \sqrt{\lambda}r$: } &&w(r) = J_{\gamma + \frac{1}{2}}(\sqrt{\lambda}r)\\
    &\text{Boundary condition: }&&J_{\gamma + \frac{1}{2}}(\sqrt{\lambda}a) = 0\\
    &\text{Since $\gamma = 0$: }&&J_{\frac{1}{2}}(\sqrt{\lambda}a) = 0\\
\end{align*}
Theorem: Let Laplacian on $S^2$ admits an ONB of eigenfunction $Y_{kj}$ such that:
\begin{align*}
    &\Delta_{\theta, \varphi}Y_{jm} = -\gamma_{jm}Y_{jm}\\
    &\text{Then: }Y_{jm} = e^{im\varphi}P_l^{|m|}\cos\theta &&\text{(Legendre polynomials)}\\
\end{align*}
Hence, solution:
\begin{align*}
    u(t, r, \theta, \varphi) &= \sum_{j,m,k}\frac{1}{\sqrt{r}}\cos\left(t\sqrt{\lambda(\gamma_{mj})}\right)J_{\gamma_{mj}}\left(r\sqrt{\lambda_k(\gamma_{mj})}\right)Y_{mj}\\
    &\quad+\sum_{j,m,k}\frac{1}{\sqrt{r}}\sin\left(t\sqrt{\lambda(\gamma_{mj})}\right)J_{\gamma_{mj}}\left(r\sqrt{\lambda_k(\gamma_{mj})}\right)Y_{mj}
\end{align*}
\begin{align*}
    \underbrace{\frac{1}{\sqrt{r}}J_{\gamma_{mj}}\left(r\sqrt{\lambda_k(\gamma_{mj})}\right)}_{\text{radial part}}\overbrace{Y_{mj}}^{\text{spherical part}} \text{ is the eigenfunction to $\Delta V = -\lambda V$}
\end{align*}\newpage
\noindent Solution to heat problem:
\begin{itemize}
    \item Problem:
    \begin{align*}
        &u_t = \Delta u\\
        &u = T(t)V(x)\\
        &T(t) = e^{-\lambda t}\\
        &\Delta V = -\lambda V\\
    \end{align*}
    \item When $\gamma = 0$:
    \begin{align*}
        &J_{\frac{1}{2}}(\sqrt{\lambda_k}a) = 0\\
        &u(x, t) = \sum_k  \frac{1}{\sqrt{r}}e^{-t\lambda k}a_kJ_{\frac{1}{2}}(\sqrt{\lambda_k}r)
    \end{align*}
\end{itemize}
\vspace{0.3em}


\subsection*{LOS 3. Learn theory about Hermite polynomials along with its properties}
Properties:
\begin{enumerate}
    \item Generating function: $e^{xz - \frac{z^2}{2}} = \sum_{n=0}^\infty P_n(x)\frac{z_n}{n!}$
    \item $P_n(x) = \left.\frac{d}{dz}^n\right\rvert_0$
    \begin{align*}
        &P_0(x) = 1 &&P_2(x) = x^2-1\\
        &P_1(x) = x &&P_3(x) = x^3-3x
    \end{align*}
    \item Let $d\mu(x) = e^{-\frac{x^2}{2}}\frac{dx}{\sqrt{2\pi}}$, then:
    \begin{align*}
        &\int_{n,m \in \mathbb{N}}P_n(x)P_m(x)d\mu(x) = n!\delta_{nm} \\
        \text{Proof: }&\int_{\mathbb{R}}e^{xz - \frac{z^2}{2}} e^{xw - \frac{w^2}{2}}d\mu(x)=\sum_{nm}\frac{z^nw^m}{n!m!}\int_{\mathbb{R}}P_n(x)P_m(x)d\mu(x)\\
        \text{LHS }&e^{-\frac{z^2}{2}}e^{-\frac{w^2}{2}}\int_{\mathbb{R}}e^{x(z+w)}e^{-\frac{x^2}{2}}\frac{dx}{\sqrt{2\pi}}=\\
        &e^{-\frac{z^2}{2}}e^{-\frac{w^2}{2}}e^{\frac{(z+w)^2}{2}} = e^{zw}\\
        \text{RHS }&\sum_{k\in\mathbb{R}}z^kw^k\frac{1}{k!}\\
        &\text{For $n\ne m$ } \int_{\mathbb{R}}P_n(x)P_m(x)d\mu(x) = 0\\
        &\text{For $n= m$ } \frac{1}{n!}\frac{1}{n!}\int_{\mathbb{R}}P_n(x)P_m(x)d\mu(x) = \frac{1}{n!}\\
    \end{align*}
    \item Previous proof implies theorem:
    \begin{align*}
        \hat{P}_j(x) = \frac{P_j(x)}{\sqrt{j!}}&\text{ ONB for }L_2(R, u) 
    \end{align*}
    \item $P'_n(x) = nP_{n-1}(x)$
    \begin{align*}
        \text{Proof: }P'(x)&=\frac{d}{dx}e^{xz - \frac{z^2}{2}}\\
        &=ze^{xz - \frac{z^2}{2}}\\
        &=\sum_{n=0}^\infty P_n(x)\frac{z^{n+1}}{n!} = nP_{n-1}(x)
    \end{align*}
    \item $P_{n+1}(x) = xP_n(x) - nP_{n-1}(x)$ or $P_n(x) = xP_{n-1}(x) - (n-1)P_{n-2}(x)$
    \begin{align*}
        \text{Proof: }&\frac{d}{dz}e^{xz - \frac{z^2}{2}} = (x-z)e^{xz - \frac{z^2}{2}}\\
        &\text{LHS}=\sum_{n=0}^\infty P_n(x)\frac{z^{n-1}}{(n-1)!} = \sum_{n=0}^\infty P_{n+1}(x)\frac{z^{n}}{n!}\\
        &\text{RHS}=x\sum_{n=0}^\infty P_n(x)\frac{z^{n}}{n!} - \sum_{n=0}^\infty P_n(x)\frac{z^{n+1}}{n!}
    \end{align*}
    \item $P_n$ are eigenfunctions for:
    \begin{align*}
        &A = -\frac{d^2}{dx^2}+x\frac{d}{dx} = -\Delta + (x, \nabla)&&\text{(Ornstein-Uhlenbek)}\\
        &AP_n(x) = nP_n(x)\\\\
        \text{Proof }&\psi(x, z) = e^{xz - \frac{z^2}{2}}\\
        &\frac{d}{dx} \psi(x, z) = z\psi(x, z)\\
        &\frac{d^2}{dx^2}\psi(x, z) = z^2\psi(x, z)\\\\
        &A = (-z^2+zx)\psi(x, z)\\
        &\quad= z(x-z)\psi(x, z)\\
        &\quad=z\frac{d}{dz}\psi(x, z)\\\\
        \text{LHS }&=\sum_{n=0}^\infty A(P_n)(x)\frac{z^n}{n!}\\
        \text{RHS }&=\sum_{n=0}^\infty \frac{P_n(x)}{n!}z(nz^{n-1})\\
        &=\sum_{n=0}^\infty nP_n(x)\frac{z^n}{n!}
    \end{align*}
    \item $A$ is self-adjoint and $(1+A)^{-1}$ is compact, therefore satisfies:
    \begin{align*}
        &(f, Ah)_\mu = \int_\mathbb{R}f'h'd\mu(x)\\
        &(f, h)_\mu = \int_\mathbb{R}\overline{f(x)}h(x)d\mu(x)\\
        \text{Proof: }&\int_\mathbb{R}f'h'e^{-\frac{x^2}{2}}\frac{dx}{\sqrt{2\pi}}=\\
        &-\int_\mathbb{R}f(h'e^{-\frac{x^2}{2}})'\frac{dx}{\sqrt{2\pi}}=\\
        &-\int_\mathbb{R}f(h''-xh')e^{-\frac{x^2}{2}}\frac{dx}{\sqrt{2\pi}}=(f, Ah)_\mu\\
    \end{align*}
    \item Solution to $A$:
    \begin{align*}
        &A = -\frac{d^2}{dx^2}+x\frac{d}{dx}\\\\
        &AP_n(x) = nP_n(x)\\
        &-P''_n + xP'_n = nP_n\\
        & 0 = P''_n - xP'_n+nP_n\\\\
        &Q_n(x) = P_n(\sqrt{2}x)&&\text{By change of variables}\\
        &Q'(x) = \sqrt{2}P'_n(\sqrt{2}x)\\
        &Q''(x) = 2P''_n(\sqrt{2}x)\\\\
        &0 = P''_n(\sqrt{2}x) - \sqrt{2}xP'_n(\sqrt{2}x)+nP_n(\sqrt{2}x)\\
        &0 = Q''_n(x) - 2xQ'_n + 2nQ_n
    \end{align*}
\end{enumerate}
\vspace{0.3em}

\subsection*{LOS 4. Solve harmonic oscillator PDE using separation ansatz and power series ansatz}
Problem:
\begin{align*}
    -uu_t = u_{xx} - x^2u
\end{align*}
Separation:
\begin{align*}
    &u(t, x) = T(t)V(x)\\
    &-iT'V = V'' - x^2T\\
    &-i\frac{T'}{T} = \frac{V''-x^2V}{V} = -\lambda\\\\
    &T' = \frac{\lambda}{i}T = -i\lambda T\\
    &T(t) = Ce^{-i\lambda t}\\\\
    &V''+(\lambda - x^2)V = 0\\
    &w(x) = e^{\frac{x^2}{2}}V(x)&& \text{By change of variable}\\
    &V = e^{-\frac{x^2}{2}}w\\
    &V' = xe^{-\frac{x^2}{2}}w + e^{-\frac{x^2}{2}}w\\
    &V'' = x^2e^{-\frac{x^2}{2}}w - 2xe^{-\frac{x^2}{2}}w' + e^{-\frac{x^2}{2}}w'' - e^{-\frac{x^2}{2}}w\\
    &\quad\; =-v + x^2v + 2xe^{-\frac{x^2}{2}}w' + e^{-\frac{x^2}{2}}w''\\\\
    &w'' - 2xw' + (\lambda - 1)w  = 0
\end{align*}
Solution:
\begin{align*}
    &w(x) = H_{k_0}(x) = \sum_k a_k x^k\\
    &(k+2)(k+1)a_{k+2}x^k - 2ka_kx^k+(\lambda -1)a_kx^k=0&&k\in\mathbb{Z}^+\\
    &a_{k+2}=\frac{2k-(\lambda -1)}{(k+2)(k+1)}a_k\\
    &\text{For terminating power series: }2k_0 = \lambda - 1\\\\
    &H_0(x) = 1&&\lambda = 1\\
    &H_1(x) = x&&\lambda = 3\\
    &H_2(x) = 4x^2-1&&\lambda = 5\\
    &\vdots
\end{align*}
Observation:
\begin{align*}
    &H_n(x) = 2^\frac{n}{2}Q_n = P_n(\sqrt{2}x)\\
    &\lambda = 2n+1 && (\text{from }2k_0 = \lambda - 1)\\
    &u(t, x) = \sum_n a_ne^{-i(2n+1)t}P_n(\sqrt{2}x)e^{-\frac{x^2}{2}}\\
    &u(0, \frac{x}{\sqrt{2}}) = \sum_n a_nP_n(x)e^{-\frac{x^2}{2}}\\
\end{align*}
The above solution applies if boundary condition satisfies $u(0, \frac{x}{\sqrt{2}})e^{-\frac{x^2}{2}} \in L_2(\gamma)$
\vspace{0.3em}


\end{document}