\documentclass[12pt, a4paper]{article}
\usepackage{titlesec}
\usepackage{lipsum}
\usepackage{amssymb}
\usepackage{amsmath}
\usepackage{bbm}
\usepackage[margin=1.3in]{geometry}
\usepackage[mathscr]{euscript}
\usepackage{tabto}
\usepackage{cancel}


\DeclareSymbolFont{rsfs}{U}{rsfs}{m}{n}
\DeclareSymbolFontAlphabet{\mathscrsfs}{rsfs}
\titlelabel{\thetitle.\quad}
\righthyphenmin=1000
\lefthyphenmin=1000


\begin{document}
\section*{Unit 7}
\vspace{1em}

\subsection*{LOS 1. Solve the heat equation using Fourier analysis}
Problem:
\begin{align*}
    &u_t = u_{xx} & 0\leq x\leq\pi\\
    &u(0,t) = h(t)\\
    &u(\pi, t) = k(t)\\
    &u(x, 0) = 0
\end{align*}
Method 1 (Fourier Analysis):
\begin{itemize}
    \item Assume $u$ is a solution and takes the form of the generic solution for periodic case:
    \begin{align*}
        &u(x, t) = \sum_{n=0}^{\infty}u_n(t)f_n(x)\\
        &f_n = \begin{cases}
            1 & n = 0\\
            \sin(nx) & n \ne 0
        \end{cases}\\\\
        &u_t(x, t) = \sum_{n=0}^{\infty}v_n(t)f_n(x)\\
        &u_{xx}(x, t) = \sum_{n=0}^{\infty}w_n(t)f_n(x)\\
    \end{align*}
    \item All derivatives are square-integrable ($\in L_2$) and therefore have unique Fourier expansion. Assuming solution is smooth ($\in L_2$), the Fourier expansions above converge pointwise
    \item Evaluate $v_n$ (assume we can interchange differentiation and integration):
    \begin{align*}
        &u_t(x, t) =\sum(f_n, u_t)f_n\\
        &\therefore v_n(t)=(f_n, u_t) = \int \bar{f_n}u_tdx\\\\
        &v_0 = \int_0^\pi u_t(x, t)\frac{dx}{\pi}= \frac{d}{dt}\int_0^\pi u(x, t)\frac{dx}{\pi}\\
        &v_n = \int_0^\pi \sin(nx)u_t(x, t)\frac{2dx}{\pi}= \frac{d}{dt}\int_0^\pi \sin(nx)u(x, t)\frac{2dx}{\pi}\\
        &\therefore v_n = \frac{d}{dt}u_n = u_n'(t)
    \end{align*}
    \item Evaluate $w_n$ using integration by parts:
    \begin{align*}
        \frac{\pi}{2}w_n &=\frac{d^2}{dx^2}\int_0^\pi\sin(nx)u(x, t)dx=\int_0^\pi \sin(nx)u_{xx}(x, t)dx\\
        \frac{\pi}{2}w_n &=\cancelto{0}{\left.\sin(nx)u_x(t)\right\rvert_0^\pi} - n\int_0^\pi \cos(nx)u_x(x, t)dx\\
        &=-n\left[\left.\cos(nx)u_x(t)\right\rvert_0^\pi\right]-n^2\int_0^\pi \sin(nx)u_x(x, t)dx\\
        &=-n[(-1)^nk(t)-h(t)]-n^2\frac{\pi}{2}u_n(t)\\
    \end{align*}
    \item Establish that $w_n = u_n'(t)$:
    \begin{align*}
        &v_n(t) = \frac{d}{dx}u_n(t) = u_t = u_n'(t)\\
        &w_n(t) = \frac{d^2}{dx^2}u_n(t) = u_xx\\
        &u_t = u_{xx}\\
        &v_n = w_n\\
        &u_n'(t)= w_n\\\\
        &\therefore u_n'(t)=\frac{2}{\pi}\left(-n[(-1)^nk(t)-h(t)]-n^2\frac{\pi}{2}u_n(t)\right) \\
        &u_n'(t)=-\frac{2n}{\pi}[(-1)^nk(t)-h(t)]-n^2u_n(t) && (1)
    \end{align*}
    \item Solve for (1) using ODE approach:
    \begin{align*}
        &u_n'(t)+n^2u_n(t)=-\frac{2n}{\pi}[(-1)^nk(t)-h(t)]\\
        &g' +n^2g = H    
    \end{align*}
    \item Solve homogeneous and then use Duhamel's principle:
    \begin{align*}
        &g' = -n^2g\\
        &S(t)(\phi)= e^{-tn^2}g(0)= e^{-tn^2}\phi\\
        &g(t) = \int_0^t S(t-s)H(s)ds + e^{-tn^2}\phi\\
        u(x, 0) = \phi = 0 \qquad &g(t) = \int_0^t S(t-s)H(s)ds + \cancelto{0}{e^{-tn^2}\phi}\\
        &u_n(t) = \frac{2n}{\pi} \int_0^t e^{-(t-s)n^2}[h(s) - (-1)^nk(s)]ds\\\\
        &u(x, t) = \sum_{n=0}^{\infty}u_n(t)f_n(x)\\
        &u(x, t) = u_0(t)f_0(x) + \sum_{n\geq 1}^{\infty}u_n(t)\sin(nx)
    \end{align*}
    \item Evaluate $u_0(t)f_0(x)$:
    \begin{align*}
        &f_0(x) = 1\\
        &u(x, t) = u_0(t) + \sum_{n\geq 1}^{\infty}u_n(t)\sin(nx) \\
        &u(0, t) = u_0(t) + \sum_{n\geq 1}^{\infty}u_n(t) \cancelto{0}{\sin(nx)} \\
        &u(0, t) = u_0(t) \rightarrow h(t) = u_0(t)\\
        &\therefore u(x, t) = h(t) + \sum_{n\geq 1}^{\infty}u_n(t)\sin(nx)
    \end{align*}
\end{itemize}
\vspace{0.3em}

\subsection*{LOS 2. Solve the heat equation using transformation onto a source equation}
Problem:
\begin{align*}
    &u_t = u_{xx} & 0\leq x\leq\pi\\
    &u(0,t) = h(t)\\
    &u(\pi, t) = k(t)\\
    &u(x, 0) = 0
\end{align*}
Method 2 (Shifting Data / Transformation onto a source equation):
\begin{itemize}
    \item Assume a solution $w$:
    \begin{align*}
        &w_t = w_{xx}\\
        &w(0, t) = h(t)\\
        &w(\pi, t) = k(t)\\
        &w(x, t) = \left(1-\frac{x}{\pi}\right)h(t) + \left(\frac{x}{\pi}\right)k(t)
    \end{align*}
    \item Assume $u$ is also a solution such that:
    \begin{align*}
        &v = u-w\\
        &v(0,t) = 0\\
        &v(\pi,t) = 0\\
        &v_{xx} = u_{xx} - w_{xx}\\
        &v_t = u_t - w_t\\
        &v_{xx} - v_t = u_{xx} - w_{xx}- (u_t - w_t)\\\\
        &w_{xx}(x, t) = 0\\
        &w_t(x, t) = \left(1-\frac{x}{\pi}\right)h'(t) + \left(\frac{x}{\pi}\right)k'(t)\\
        &\therefore v_{xx} - v_{t} = w_t\\
        &v(x, 0) = \cancelto{0}{u(x, 0)}- w(x, 0) = -\left(1-\frac{x}{\pi}\right)h(0) - \left(\frac{x}{\pi}\right)k(0)= \phi(x)
    \end{align*}
    \item In general:
    \begin{align*}
        &v_t = L(v)\\
        &v_t - v_{xx} = g(x, t)\\
        &v(x, 0) = \phi(x)\\\\
        &v(x, t) =  \int_0^tS(t-s)g(s)ds + S(t)(\phi)\\
        &\qquad\int_0^tS(t-s)g(s)ds= \int_0^t\left(\hat{g}_0(x, s)+\sum_{n\geq 1}^\infty e^{-(t-s)n^2}\hat{g}_n(x, s)\sin(nx)\right)ds\\\\
        &\qquad \hat{g}_n(x, s) = \frac{2}{\pi} \int_0^\pi \sin(nx)g(x, s)dx\\
        &\qquad \hat{g}_0(x, s) = \frac{1}{\pi} \int_0^\pi g(x, s)dx\\\\
        &\qquad S(t)(\phi) = \tilde{\phi}_0 + \sum_{n\geq 1}e^{-tn^2}\tilde{\phi}_n\sin(nx)\\
        &\qquad \tilde{\phi}_n(x) = \frac{2}{\pi} \int_0^\pi \sin(nx)\phi(x)dx\\
        &\qquad \tilde{\phi}_0(x) = \frac{1}{\pi} \int_0^\pi \phi(x)dx\\\\
    \end{align*}
    \item Combining the solutions:
    \begin{align*}
        v(x, t) &=  \tilde{\phi}_0 + \sum_{n\geq 1}e^{-tn^2}\tilde{\phi}_n\sin(nx) + \sum_{n\geq 1}^{\infty}\int_0^te^{-(t-s)n^2}\hat{g}_n(x, s)ds\sin(nx)\\
        &= \tilde{\phi}_0 + \sum_{n\geq 1}e^{-tn^2}\tilde{\phi}_n\sin(nx) + \sum_{n\geq 1}^{\infty}v_n(t)\sin(nx)
    \end{align*}
    \item Check Page 151
\end{itemize}
\vspace{0.3em}

\subsection*{LOS 3. Solve homogeneous heat equation in higher dimensions using Fourier analysis}
Problem
\begin{align*}
    &\Delta f = \sum_{k=1}^n\frac{d^2}{dx_k^2}f\\
    &u_t= \Delta u\\
    &u(0, x) = \phi(x)
\end{align*}
Facts:
\begin{itemize}
    \item Fourier transform formula:
    \begin{align*}
        &\hat{f}(\xi) = \int_{\mathbb{R}}e^{-i(\xi, x)}f(x)\frac{dx}{\sqrt{2\pi}^n}
    \end{align*}
    \item Dot product:
    \begin{align*}
        &(\xi, x) = \sum_{j=1}^n\xi_jx_j
    \end{align*}
    \item Lemma 1:
    \begin{align*}
        &f(x) = e^{-\frac{|x^2|}{2}}\\
        &|x| = (x, x)^{\frac{1}{2}}\\
        &\hat{f}(\xi) = e^{-\frac{|\xi^2|}{2}}
    \end{align*}
    \item Lemma 2 (Parseval):
    \begin{align*}
        \|f\|_{L_2}=\|\hat{f}\|_{L_2}
    \end{align*}
    \item Lemma 3 (Fourier inversion - applies when $f$ is sufficiently smooth and decays when approaching infinity):
    \begin{align*}
        f(x) = \int_{\mathbb{R}}e^{-i(\xi, x)}\hat{f}(\xi)\frac{d\xi}{\sqrt{2\pi}^n}
    \end{align*}
    \item Lemma 4:
    \begin{align*}
        &\widehat{\frac{d}{dx_k}f} = i\xi_k\hat{f}(\xi)\\
        &\hat{u}_t(t, \xi) = \widehat{\frac{d}{dt}u}(t, \xi) = \widehat{\frac{d^2}{dx^2}u}(t, \xi) = -|\xi|^2\hat{u}(t,\xi)\\
        &\hat{u}(t, \xi) = e^{-t|\xi|^2}\hat{u}(0, \xi) = e^{-t|\xi|^2}\hat{\phi}(\xi)
    \end{align*}
    \item Therefore, for u: $\mathbb{R}\times\mathbb{R}^n$ ($\mathbb{R}$ for $t$ space and $\mathbb{R}^n$ for $x$ dimension):
    \begin{align*}
        \hat{u}(t, \xi) = \int_{\mathbb{R}}e^{-i(\xi, x)}u(t, x)\frac{d\xi}{\sqrt{2\pi}^n}
    \end{align*}
    \item From Lemma 1:
    \begin{align*}
        e^{-t|\xi|^2} = e^{-\frac{|\sqrt{2t}\xi|^2}{2}} = \int_{\mathbb{R}^n}e^{i(-\sqrt{2t}\xi, x)}e^{-\frac{|x|^2}{2}}\frac{dx}{\sqrt{2\pi}^n}
    \end{align*}
    \item By change of variable:
    \begin{align*}
        &y = \sqrt{2t}x\\
        &dy = \sqrt{2t}^ndx\\
        &\therefore e^{-t|\xi|^2} = \int_{\mathbb{R}^n}e^{-i(\xi, y)}e^{-\frac{|y|^2}{4t}}\frac{dx}{\sqrt{4\pi t}^n}\\\\
        u(t, x)&=\int_{\mathbb{R}^n}e^{i(\xi, x)}\hat{u}(t, \xi)\frac{d\xi}{\sqrt{2\pi}^n}\\
        &=\int_{\mathbb{R}^n}e^{i(\xi, x)}e^{-t|\xi|^2}\hat{\phi}(\xi)\frac{d\xi}{\sqrt{2\pi}^n}\\
        &=\int_{\mathbb{R}^n}e^{i(\xi, x)} \int_{\mathbb{R}^n}e^{-i(\xi, y)}e^{-\frac{|y|^2}{4t}}\frac{dx}{\sqrt{4\pi t}^n} \hat{\phi}(\xi)\frac{d\xi}{\sqrt{2\pi}^n}\\
        &=\int_{\mathbb{R}^n}\int_{\mathbb{R}^n}e^{i(\xi, x)} e^{-i(\xi, y)} \hat{\phi}(\xi)\frac{d\xi}{\sqrt{2\pi}^n}e^{-\frac{|y|^2}{4t}}\frac{dx}{\sqrt{4\pi t}^n}\\
        &=\int_{\mathbb{R}^n}\phi(x-y)e^{-\frac{|y|^2}{4t}}\frac{dx}{\sqrt{4\pi t}^n}\\
    \end{align*}
\end{itemize}

\vspace{0.3em}

\subsection*{LOS 4. Solve nonhomogeneous heat equation in higher dimensions using Fourier analysis}
Problem:
\begin{align*}
    &u_t - \Delta u = w\\
    &u(0, x) = \phi(x)
\end{align*}
Homogeneous solution:
\begin{align*}
    &S(t)(\phi)(x) = \int_{\mathbb{R}^n}\phi(x-y)e^{\frac{-|y|^2}{4t}}\frac{dy}{\sqrt{4\pi t}^n}\\
\end{align*}
Inhomogeneous solution (using Duhamel):
\begin{align*}
    &u(t, x) = S(t)(\phi)(x) + \int_0^tS(t-s)w(s)ds
\end{align*}
\vspace{0.3em}

\subsection*{LOS 5. Understand the polar coordinates in higher dimensions}
Problem:
\begin{align*}
    &u_{tt} = \Delta u\\
    &u(0, x) = \phi(x)\\
    &u_t(0, t) = \psi(x)\\
\end{align*}
From LOS 6:
\begin{align*}
    \hat{u}(t, \xi) &= cos(t|\xi|)\hat{\phi}(\xi) + \frac{\sin(t|\xi|)}{|\xi|}\hat{\psi}(\xi)\\
    &=\widehat{u_\phi}(t, \xi) + \widehat{u_\psi}(t, \xi)\\
\end{align*}
Remark:
\begin{align*}
    \frac{d}{dt}\hat{u}_\psi(t, \xi)=  cos(t|\xi|)\hat{\phi}(\xi)\\
\end{align*}
Example in 1-D (shows that we can focus on getting $u_\psi$ only to get the full solution):
\begin{align*}
    u(t, x)&=\frac{\phi(x+t)+\phi(x-t)}{2} +\frac{1}{2}\int_{x-t}^{x+t}\phi(x, s)ds\\
    &=u_\phi+u_\psi\\\\
    \frac{d}{dt}u_\psi(t, x)&= \frac{\psi(x+t)+\psi(x-t)}{2} \rightarrow \text{ same form as Dirichlet solution}\\
\end{align*}
Convolution:
\begin{align*}
    &f*g(x) = \int_{\mathbb{R}^n}f(x-y)g(y)dy
\end{align*}
Lemma 1:
\begin{itemize}
    \item Claim:
    \begin{align*}
        \widehat{f*g}(\xi) = \sqrt{2\pi}^n\hat{f}(\xi)\hat{g}(\xi)
    \end{align*}
    \item Proof:
    \begin{align*}
        \widehat{f*g}(\xi) &= \int_{\mathbb{R}^n}e^{-i(\xi, x)}f*g(x)\frac{dx}{\sqrt{2\pi}^n}\\
        &= \int_{\mathbb{R}^n}\int_{\mathbb{R}^n}e^{-i(\xi, x)}f(x-y)g(y)dy\frac{dx}{\sqrt{2\pi}^n}\\
        &= \int_{\mathbb{R}^n}\int_{\mathbb{R}^n}e^{-i[(\xi, x-y)+(\xi, y)]}f(x-y)g(y)dy\frac{dx}{\sqrt{2\pi}^n}\\
        &= \sqrt{2\pi}^n\int_{\mathbb{R}^n}\int_{\mathbb{R}^n}e^{-i(\xi, x-y)}f(x-y)\frac{dy}{\sqrt{2\pi}^n}\; e^{-i(\xi, y)}g(y)\frac{dx}{\sqrt{2\pi}^n}\\\\
        &z = x-y\\
        &= \sqrt{2\pi}^n\int_{\mathbb{R}^n}\int_{\mathbb{R}^n}e^{-i(\xi, z)}f(z)\frac{dz}{\sqrt{2\pi}^n}\; e^{-i(\xi, y)}g(y)\frac{dx}{\sqrt{2\pi}^n}\\
        &= \sqrt{2\pi}^n\hat{f}(\xi)\hat{g}(\xi)
    \end{align*}
    \item Apply the lemma to $u_\psi$:
    \begin{align*}
        u_\psi(t, x) = \sqrt{2\pi}^n \hat{g}^t*\psi\\
    \end{align*}
\end{itemize}
Lemma 2:
\begin{itemize}
    \item $F:\mathbb{R}^n \rightarrow \mathbb{R}$ is radial if $\exists H:\mathbb{R} \rightarrow \mathbb{R}$ such that $F(x) = H(|x|)$
    \item Polar coordinates:
    \begin{align*}
        \int_{\mathbb{R}^n}f(x)dx = C_n\int_0^\infty\int_{S^{n-1}}f(ry)d\sigma(y)r^{n-1}dr
    \end{align*}
    \item Claim: if $F$ is radial, then $\hat{F}$ is also radial
    \begin{align*}
        \hat{F}(\xi)&=\int_{\mathbb{R}^n}e^{-i(\xi, y)}F(x)\frac{dx}{\sqrt{2\pi}^n}\\
        &=C_n\int_0^\infty\left[\int_{S^{n-1}}e^{-i(r\xi, y)}d\sigma(y)\right]F(x)r^{n-1}dr \qquad \text{where }x=ry\\
        &=C_n\int_0^\infty\left[\int_{S^{n-1}}e^{-i(r\xi, y)}d\sigma(y)\right]H(r)r^{n-1}dr \qquad \text{where }r=|x|\\
        V(r|\xi|)&=\int_{S^{n-1}}e^{-i(r\xi, y)}d\sigma(y) \rightarrow \text{depends only on $|\xi|$, therefore radial}
    \end{align*}
    \item $\sigma$ is the probability measure on $S^{n-1} = \{y\;|\;|y|=1\}$ which is invariant under rotation (periodic)\\
    \item Simple example:
    \begin{align*}
        F(x) = \begin{cases}
            1&|x|\leq R\\
            0&|x|\>R
        \end{cases}\\\\
        \text{Volume}_n(\text{Ball with radius $R$}) &=C_n\int_0^Rr^{n-1}dr\\
        &=\frac{C_n}{n}R^n 
    \end{align*}
    \item Example for $n=2$:
    \begin{align*}
        (x, y)&= re^{i\theta}\\
        F(x, y) &= H(r)\\\\
        S^{n-1} &= \{e^{i\theta}\;|\; 0 \leq \theta \leq 2\pi\}\\
        d\sigma(\theta)&=\frac{d\theta}{2\pi}\\\\
        \xi&\in\mathbb{R}^2\\
        \xi&=re^{i\eta}\\
        (re^{i\eta}, e^{i\theta}) &= (r, e^{i(\theta-\eta)})\\\\
        \hat{F}(\xi)&=\int_0^{2\pi}e^{-i(r, e^{i(\theta-\eta)})}\frac{d\theta}{2\pi}\\
        &=\frac{1}{2\pi}\int_0^{2\pi}\cos(r\cos\theta)d\theta\\
        &=V_2(r) \qquad \text{ where }r=|\xi|
    \end{align*}
\end{itemize}
Lemma 3 (Dilation):
\begin{itemize}
    \item Claim, for $f:\mathbb{R}^n\rightarrow\mathbb{R}$:
    \begin{align*}
        \text{If }&f^t(x)=f(tx)\\
        &\hat{f}^t(\xi)=t^{-n}\hat{f}\left(\frac{\xi}{t}\right) 
    \end{align*}
    \item Proof:
    \begin{align*}
        \hat{f}^t(\xi)&=\int_{\mathbb{R}^n}e^{-i(\xi, x)}f^t(x)\frac{dx}{\sqrt{2\pi}^n}\\
        &=\int_{\mathbb{R}^n}e^{-i(\xi, x)}f(tx)\frac{dx}{\sqrt{2\pi}^n}\\\\
        &y = tx\\
        &dy = t^ndx\\
        &=t^{-n}\int_{\mathbb{R}^n}e^{-i(\frac{\xi}{t}, y)}f(y)\frac{dy}{\sqrt{2\pi}^n}\\
        &=t^{-n}\hat{f}\left(\frac{\xi}{t}\right)
    \end{align*}
    \item Apply to $\hat{g}^t$:
    \begin{align*}
        \hat{g}^t(\xi) & = \frac{\sin(t|\xi|)}{|\xi|}= \frac{\sin(t|\xi|)}{|\xi|}\frac{t}{t}\\\\
        & s = \frac{1}{t}\\
        &= \frac{1}{s}\frac{\sin(\frac{|\xi|}{s})}{\frac{|\xi|}{s}}=s^{n-1}s^{-n}\frac{\sin(\frac{|\xi|}{s})}{\frac{|\xi|}{s}}\\\\
        \hat{g}^t(\xi) &= t^{1-n}\hat{g}^1\left(\frac{\xi}{t}\right)\qquad \text{where }\hat{g}^1(\xi) = \frac{\sin(|\xi|)}{|\xi|}
    \end{align*}
\end{itemize}
Apply the two lemmas to the original problem:
\begin{align*}
    u_\psi(t, 0) &= C_n\int_{\mathbb{R}^n} g^t(x-y)\psi(y)dy\\
    &=C_n t^{n-1}\int_{\mathbb{R}^n} g^1(\frac{y}{t})\psi(y)dy\\
    &=C_n(t) \int_{\mathbb{R}^n}\left[\int_{S^{n-1}}\psi(ry)d\sigma(y)\right]g(t, r)r^{n-1}dr\\
\end{align*}
Conclusion: for any dimension, solution at the space is the overage of all circles, then apply a scalar factor to the radii at every t
\vspace{0.3em}

\subsection*{LOS 6. Determine solution of wave equation using Fourier transforms}
Problem:
\begin{align*}
    &u_{tt} = \Delta u\\
    &u_t(0, x) = \phi(x)\\
    &u_t(0, x) = \psi(x)\\
    &\widehat{u_{tt}} = \widehat{\Delta u} = \widehat{\frac{d^2}{dx^2}u} = -|\xi|^2\hat{u}(t, \xi)
\end{align*}
Assume an ODE problem:
\begin{align*}
    &f''(t) = -\alpha f(t) && \alpha > 0\\
    &f(t) = a(0)\cos(\sqrt{a}t) + b(0)\sin(\sqrt{a}t)
\end{align*}
Translate to the original problem:
\begin{align*}
    &\hat{u}(t, \xi) = \cos(t|\xi|)a(\xi) + \sin(t|\xi|)b(\xi)\\
    &\hat{u}(0, \xi) = a(\xi) = \hat{\phi}(\xi)\\
    &\hat{u}_t(0, \xi) = |\xi|b(\xi) = \hat{\psi}(\xi)\\
    &\qquad b(\xi) = \frac{\psi(\xi)}{|\xi|}\\\\
    &S_D(t) = \int_{\mathbb{R}^n}e^{i(\xi, x)}\cos(|\xi|t)\hat{\phi}(\xi)\frac{d\xi}{\sqrt{(2\pi)}^n}\\
    &S_N(t) = \int_{\mathbb{R}^n}e^{i(\xi, x)}\sin(|\xi|t)\frac{\hat{\psi}(\xi)}{|\xi|}\frac{d\xi}{\sqrt{(2\pi)}^n}\\
\end{align*}
In wave equation, the following equation holds:
\begin{align*}
    &S_D(0) = \text{Id} && S'_D(0) = 0\\
    &S_N(0) = 0 && S'_N(0)=\text{Id}
\end{align*}
Remarks:
\begin{align*}
    \int_0^t \cos(|\xi|s)ds = \frac{sin(|\xi|t)}{|\xi|}
\end{align*}
Example for 1-D ($n=1$):
\begin{align*}
    \cos(|\xi|t) &= \frac{e^{it\xi} - e^{-it\xi}}{2}\\
    S_D(t) &= \frac{1}{2}\int_{\mathbb{R}}e^{i\xi x}e^{i\xi t}\hat{\phi}(\xi)\frac{d\xi}{\sqrt{2\pi}}+\frac{1}{2}\int_{\mathbb{R}}e^{i\xi x}e^{-i\xi t}\hat{\phi}(\xi)\frac{d\xi}{\sqrt{2\pi}}\\
    &=\frac{1}{2}\left(\phi(x+t)+\phi(x-t)\right)\\
    S_N(t) &= \int_{\mathbb{R}}e^{i(\xi, x)}\sin(|\xi|t)\frac{\hat{\psi}(\xi)}{|\xi|}\frac{d\xi}{\sqrt{(2\pi)}}\\
    &=\int_{\mathbb{R}}e^{i(\xi, x)}\int_0^t \cos(|\xi|s)ds\;\hat{\psi}(\xi)\frac{d\xi}{\sqrt{(2\pi)}}\\
    &=\int_0^t\int_{\mathbb{R}}e^{i(\xi, x)} \cos(|\xi|s)\hat{\psi}(\xi)\frac{d\xi}{\sqrt{(2\pi)}}ds\\
    &=\int_0^tS_D(s)ds\\
    &=\int_0^t\frac{\psi(x+s)+\psi(x-s)}{2}ds
\end{align*}
\vspace{0.3em}

\subsection*{LOS 7. Determine solution of wave equation using polar coordinates}
Problem:
\begin{align*}
    &u_{tt} = \Delta u\\
    &u(0, x) = \phi(x)\\
    &u_t(0, x) = \psi(x)
\end{align*}
Solution by polar coordinates:
\begin{align*}
    &u_\psi(t, x) =\int g(t, |x-y|)\psi(y)dy\\
    &u_\psi(t, x) =\int_0^\infty g(t, r)\int_{S^{n-1}}\psi(ry)d\sigma(y)dr 
\end{align*}
\vspace{0.3em}

\subsection*{LOS 8. Find the solution of wave equation using spherical means method}
By translation:
\begin{itemize}
    \item Assume $u$ is a solution
    \item For $x_0 \in \mathbb{R}^2$, the following is also a solution:
    \begin{align*}
        w(t, x) = u(t, x-x_0)
    \end{align*}
    \item Hence, we can shift our original wave problem such that:
    \begin{align*}
        &u_{tt} = \Delta u && (1)\\
        &u(0, x) = \phi(x) = 0\\
        &u_t(0, x) = \psi(x)
    \end{align*}
    \item After solving the above problem, we can shift it back to the original problem\\
\end{itemize}
Spherical means:
\begin{itemize}
    \item Let $u(t, x)$ be a solution:
    \begin{align*}
        &\bar{u}(t, x) = \int_{S^{n-1}}u(t, |x|y)d\sigma(y)\\
        &|x| = r
    \end{align*}
    \item $u$ above is only dependent on the absolute value of $x$ such that:
    \begin{align*}
        &x \ne x'\\
        \text{if }&|x| = |x'| \rightarrow \bar{u}(t, x) = \bar{u}(t, x')
    \end{align*}
    \item $u$ over the domain of $t$ is a series of circles with increasing radii as $t$ goes up. As such, the integral is simply an average of all circles over the domain of $t$.
    \item Hence:
    \begin{align*}
        \text{Given }&g(t, r) = \int_{S^{n-1}}u(t, ry)d\sigma(y)\\
        &u_\psi(t, 0) = \int_0^\infty g(t, r)\bar{\psi}(r)dr
    \end{align*} 
\end{itemize}
Lemma 1:
\begin{itemize}
    \item If $u$ solves the wave problem (1):
    \begin{align*}
        &\bar{u}_{tt} = \bar{u}_{rr} + \frac{n-1}{r}\bar{u}_r\\
        &u(t, 0) \text{ only depends on } \bar{u}
    \end{align*}
    \item Proof:
    \begin{align*}
        \bar{u}_{tt}(t, x) &= \int_{S^{n-1}}u_{tt}(t, ry)d\sigma(y)\\
        &= \int_{S^{n-1}}(\Delta u)(t, ry)d\sigma(y)\\
        &=\Delta\int_{S^{n-1}}u(t, ry)d\sigma(y)
    \end{align*}
    \item Let $G(t, r)$ be a radial function:
    \begin{align*}
        \frac{\partial}{\partial x_k}G(t, |x|) &= G_r(t, |x|)\frac{\partial}{\partial x_k}|x|\qquad\text{where }|x| = \left(\sum_{j=1}^n x_j^2\right)^{\frac{1}{2}}\\
        &= G_r(t, |x|)\frac{1}{2}\frac{2x_k}{\sqrt{\sum x_j^2}}\\
        &= G_r(t, |x|)\frac{x_k}{|x|}\\\\
        \frac{\partial^2}{\partial x_k^2}G(t, |x|) &= G_{rr}(t, |x|)\frac{x_k^2}{|x|^2}+G_{r}(t, |x|)\frac{|x|-\frac{x_k^2}{|x|^2}}{|x|^2}\\\\
        \therefore \Delta G(t, |x|) &= \sum \frac{\partial^2}{\partial x_k^2}G(t, |x|) \\
        &=G_{rr}(t, |x|) + \frac{n}{|x|}G_r(t, |x|) - \frac{1}{|x|}G_r(t, |x|)\\
        &=G_{rr}(t, r) + \frac{n-1}{r}G_r(t, r)
    \end{align*}
\end{itemize}
Example for $n=3$:
\begin{itemize}
    \item Conversion into polar coordinates:
    \begin{align*}
        &\bar{u}_{tt} = \bar{u}_{rr} + \frac{2}{r}\bar{u}_r\\
        \text{Let }&v(t, r) = r\bar{u}(t, r)\\
        &v_{tt} = r\bar{u}_{tt}(t, r) = r[\bar{u}_rr+\frac{2}{r}\bar{u}_r]=r\bar{u}_{rr}+2\bar{u}_r\\
        &v_r = \bar{u}_r + r\bar{u}_r\\
        &v_{rr} = \bar{u}_r + \bar{u}_r + r\bar{u}_{rr} = 2\bar{u}_r + r\bar{u}_{rr} = v_{tt}
    \end{align*}
    \item New converted problem:
    \begin{align*}
        &v_{tt}=v_{rr}\\
        &v(0, r) = r\bar{u}(0, r) = r\bar{\phi} = 0\\
        &v_t(0, r) = r\bar{u}_t(r, 0) = r\bar{\psi}(r)
    \end{align*}
    \item Solution:
    \begin{align*}
        v(r, t) &= \frac{1}{2}\int_{r-t}^{r+t}s\bar{\psi}(s)ds\\
        \bar{u}(r, t) &= \frac{1}{2r}\int_{r-t}^{r+t}s\bar{\psi}(s)ds\\\\
        \bar{u}(0, t) &= \lim_{r\to 0}\frac{v(r, t)}{r}\\
        &=\left.\frac{\partial v}{\partial r}(r, t)\right\rvert_{r=0}\\
        &=\frac{1}{2}[(r+t)\bar{\psi}(r+t)+(r-t)\bar{\psi}(r-t)]\\
        &=t\bar{\psi}(t)\\\\
        \therefore u(0, t) &= t\int_{S^{n-1}}\psi(ry)d\sigma(y)
    \end{align*}
    \item Alternative solution:
    \begin{align*}
        u_\psi(x, t) &= t \int_{S^{3-1}} \psi(ty)d\sigma(y)\\
        u_\phi(x, t) &= \frac{\partial}{\partial t}\left[t \int_{S^{3-1}} \phi(ty)d\sigma(y)\right]\\
        &=\int_{S^{3-1}} \phi(ty)d\sigma(y)+t\int_{S^{3-1}} \nabla \phi(ty)yd\sigma(y)\\
        \text{Note: } \frac{\partial}{\partial t}\phi(ty_1, ty_2, ty_3) &= \nabla \phi(ty)\cdot y\\
        &=\sum_{j=1}^3 \frac{\partial}{\partial x_j}\phi(ty_j)y_j
    \end{align*}
\end{itemize}
Practical solution for $n=3$:
\begin{itemize}
    \item Convert into spherical coordinates:
    \begin{align*}
        w(\theta, \eta)&=(\sin\theta\sin\eta, \sin\theta\cos\eta, \cos\theta)\\
        |w(\theta, \eta)|^2&=1\\\\
        u(x, t) &= t\int_{S^{3-1}}\psi(ty)d\sigma(y)\\
        &= t\int_0^\pi\int_0^{2\pi}\psi(\sin\theta\sin\eta, \sin\theta\cos\eta, \cos\theta)\frac{d\eta}{2\pi} \frac{d\theta}{\pi}\\
    \end{align*}
    \item Change of variable:
    \begin{align*}
        &s = \sin\theta\\
        &\frac{ds}{d\theta} = \cos\theta = \sqrt{1-\sin^2\theta}\\
        &d\theta = \frac{ds}{\sqrt{1-\sin^2\theta}} = \frac{ds}{\sqrt{1-s^2}}\\
        &u(x_1, x_2, t) = \frac{t}{\pi} \int^{\sin\frac{\pi}{2}=1}_0\int^{2\pi}_{0}\phi(ts\sin\eta, ts\cos\eta)\frac{d\eta}{2\pi}\frac{ds}{\sqrt{1-s^2}} 
    \end{align*}
    \item Solution for problem with source:
    \begin{align*}
        &u(x, t) = \int^{t-s}_0S(t-s)g(s)ds\\
        &\text{where }s(t) = t\int_{S^{n-1}}\phi(ty)d\sigma(y)\\
    \end{align*}
\end{itemize}
Example for $n=5$:
\begin{align*}
    &&&\bar{u}_{tt} = \bar{u}_{rr} + \frac{4}{r}\bar{u}_r\\
    &\text{Let }&&v(t, r) = \left(\frac{1}{r}\frac{\partial}{\partial r}\right)(r^3\bar{u})\\
    &\text{Lemma }&&\frac{\partial^2}{\partial r^2}\left(\frac{1}{r}\frac{\partial}{\partial r}\right)^{k-1}(r^{2k-1}\phi)=\left(\frac{1}{r}\frac{\partial}{\partial r}\right)^{k}(r^{2k}\phi_r)\\\\
    &&&\frac{\partial^2}{\partial r^2}v(t, r) = \frac{\partial^2}{\partial r^2}\left(\frac{1}{r}\frac{\partial}{\partial r}\right)(r^3\bar{u})\\
    &&&=\left(\frac{1}{r}\frac{\partial}{\partial r}\right)^2(r^4\bar{u}_r)\\
    &&&=\left(\frac{1}{r}\frac{\partial}{\partial r}\right)\frac{1}{r}(4r^3\bar{u}_r+r^4\bar{u}_{rr})\\
    &&&=\left(\frac{1}{r}\frac{\partial}{\partial r}\right)\left[\frac{1}{r}(4\frac{\bar{u}_r}{r}+\bar{u}_{rr})r^3\right]\\
    &&&=\left(\frac{1}{r}\frac{\partial}{\partial r}\right)(r^3\bar{u}_{tt}) = v_{tt}
\end{align*}
Generally:
\begin{align*}
    \text{For }&n=2k-1\\
    &v(t, r)=\left(\frac{1}{r}\frac{\partial}{\partial r}\right)^{k-1}(r^{2k-1}\bar{u})\\
\end{align*}
Descent: 
\begin{itemize}
    \item We can solve even numbered dimension using odd numbered dimension problem
    \item Assume $u(x_1, x_2, t)$ is a solution to the $u_{tt}=\Delta u$
    \item Consider $\tilde{u}(x_1, x_2, x_3, t) = u(x_1, x_2, t)$ on $\mathbb{R}^3$
    \item Solve for $\tilde{u}_{tt} = \Delta_{\mathbb{R}^3} \tilde{u}$
\end{itemize}
\vspace{0.3em}

\subsection*{LOS 9. Understand the concept of special relativity}
Concepts:
\begin{itemize}
    \item Assume the speed of light $c=1$
    \item Light cone $\{(x, v) |\; |x| \leq |v|\}$
    \item The characteristic surfaces are the only one which propagate singularities
    \item Light ray $(x, v_0)={x_0+tv_0|\;|v_0|=1}$
    \item Characteristic surface is a union of light rays\\
\end{itemize}

\vspace{0.3em}

\subsection*{LOS 10. Solve the Schrödinger equation}
Problem:
\begin{align*}
    &u_t = i\Delta u\\
    &u(0, x) = \phi(x)\\
\end{align*}
Solution:
\begin{align*}
    &\hat{u}_t = -i|\xi|^2\hat{u}\\
    &\hat{u}(t, \xi) = e^{-it|\xi|^2}\hat{\phi}(\xi)\\
    &u(t, x) = \int e^{i(\xi, x)}e^{-it|\xi|^2}\hat{\phi}\frac{d\xi}{\sqrt{2\pi}^n}
\end{align*}
Remarks:
\begin{itemize}
    \item $\phi \in L_2$
    \item $\|\hat{u}(t, \xi)\|^2_{L_2} = \|\hat{\phi}\|^2_{L_2}$ energy is preserved, not decaying
    \item $\|u(t, \xi)\|^2_{L_2} = \|\phi\|^2_{L_2}$ solution is radial, therefore convolution applies\\
\end{itemize}
Generically:
\begin{align*}
    u(x, t)&= \int\phi(x-y)g\left(\frac{y}{\sqrt{2t}}\right)\frac{dy}{\sqrt{2\pi t}^n}\\
    &= \int\phi(x-y)e^{\frac{-|x-y|^2}{4t}}\frac{dy}{\sqrt{2\pi t}^n}
\end{align*}
\vspace{0.3em}

\subsection*{LOS 11. Understand the idea of self-adjoint operator for the Laplace operator}
Theorem:
\begin{itemize}
    \item Applies for a compact domain $\bar{D}$
    \item The Laplace operation with Robin boundary condition is self-adjoint
    \begin{align*}
        &\int_D\Delta uv dx = \int_D u \Delta vdx
    \end{align*}
    \item Corollary: if we have a self-adjoint operator, then we have discrete eigenvalues
    \begin{align*}
        \exists \lambda_n \text{ increasing and }V_n \in \text{ONB such that }\Delta(V_n)=-\lambda_nV_n \text{ on }D
    \end{align*}
    \item Proof:
    \begin{align*}
        &\text{Let $u$ and $v$ satisfy the boundary conditions}\\\\
        &\text{From Robin condition:}\\
        &\qquad\qquad au+\frac{\partial u}{\partial n} = 0\\
        &\qquad\qquad \frac{\partial u}{\partial n} = -au\\\\
        &\int_{D} (\Delta u)v - (\Delta v)u dx = \int_{\partial D}\left(\frac{\partial u}{\partial n}v-\frac{\partial v}{\partial n}u \right)dS \\
        &\qquad\qquad\qquad\qquad\quad\;=\int_{\partial D}(-auv+avu)dS=0\\
    \end{align*}
\end{itemize}
\vspace{0.3em}

\subsection*{LOS 12. Understand the divergence theorem and Green's Identity}
Requirement:
\begin{itemize}
    \item Compact, orientable manifold i.e. circle, sphere, cube
    \item Domain $\Omega \subset \mathbb{R}^2/\mathbb{R}^3$
    \item Boundary $\partial\Omega$ i.e. perimeter of the circle or the faces of the cube
    \item Normal vector $n$ on $\partial\Omega$ where $\|n\|=1$
    \item Vector field $O_\text{open} \supseteq \partial\Omega$\\
\end{itemize}
Theorem:
\begin{align*}
    &\int_\Omega \text{div}(F)dV(x) = \int_{\partial\Omega} F\cdot\mathbf{n}\;dS\\
    &\qquad F: O \rightarrow \mathbb{R}^n \qquad F = \left(\begin{array}{c}
        F_1\\
        \vdots\\
        F_n\\
    \end{array}\right)\\
    &\qquad V: \text{volume i.e. $dx_1 dx_2$ for circle, $dx_1dx_2dx_3$ for cube}\\
    &\qquad \mathbf{n} : \text{normal vector}\\
    &\qquad \text{div}(F) = \sum_{k=1}^n\frac{\partial}{\partial x_k}F_k\\
\end{align*}
Application:
\begin{align*}
    &u_{tt} = \Delta u &&\text{(wave)}\\
    &u_t = \Delta u &&\text{(heat)}\\
    &\left.u\right\rvert_{\partial\Omega} = 0&&(D)\\
    &\left.\frac{\partial u}{\partial n}\right\rvert_{\Omega} = 0&&(N)\\
    &au + \frac{\partial u}{\partial n} = 0&&(R)\\\\
    &\frac{\partial u}{\partial n} = \nabla \cdot \mathbf{n}&&\mathbf{n}:\text{ normal vector}\\
\end{align*}
Corollary: Green's idendity
\begin{align*}
    &\text{$u$, $v$ two functions defined on $\Omega$ and differentiable}\\\\
    &F=(\nabla u)v \qquad H=(\nabla v)u\\
    &\text{$F$, $H$ are vector fields (gradient times the function)}\\\\
    &\text{div}(F) = \sum_{k=1}^{n}\frac{\partial }{\partial k}\left(\left(\frac{\partial}{\partial x_k}u\right)v\right)\\
    &\qquad\quad =(\Delta u)v + (\nabla u, \nabla v)\\
    &\qquad\quad =\sum\left(\frac{\partial}{\partial x_k}u\right)v+\sum \frac{\partial u}{\partial x_k} \frac{\partial v}{\partial x_k}\\\\
    &\text{div}(F) - \text{div}(H)=(\Delta u)v + (\nabla u, \nabla v) - (\Delta v)u - (\nabla v, \nabla u)\\
    &\qquad\qquad\qquad\quad\; = (\Delta u)v - (\Delta v)u\\\\
    &\text{Green's Identity: } \int_{\Omega} (\Delta u)v - (\Delta v)u dx = \int_{\partial\Omega} \left(\frac{\partial u}{\partial n}v-\frac{\partial v}{\partial n}u \right)dS\\
\end{align*}

\subsection*{LOS 13. Understand the method of separation of variables for heat and wave equation}
Problem (Heat):
\begin{align*}
    u_t = \Delta u
\end{align*}
Separation Ansatz:
\begin{align*}
    &u(t, x) = T(t)V(x)\\
    &u_t(t, x) = T'(t)V(x)\\
    &\Delta u(t, x) = T(t)(\Delta V)(x)\\\\
    &\frac{T'}{T}=\frac{\Delta V}{V} = \frac{u_t}{u} = \frac{\Delta u}{u} = \frac{\Delta V_n}{V_n} = -\lambda_n\\
    &u(x, t) = \sum_n e^{-t\lambda_n}a_nV_n(x)\\
    &a_n = \int_D \bar{V}_n \phi(x)dx\\
\end{align*}
Problem (Wave):
\begin{align*}
    &u_{tt} = \Delta u\\
    &\frac{T''}{T}=\frac{\Delta V}{V} = -\lambda_n\\
    &T(t) = \cos(\sqrt{\lambda_n}t)+\sin(\sqrt{\lambda_n}t)\\
    &u(t, x) =  \sum_na_n\cos(\sqrt{\lambda_n}t)V_n(x)+\sum_nb_n\sin(\sqrt{\lambda_n}t)V_n(x)
\end{align*}
\vspace{0.3em}



\end{document}