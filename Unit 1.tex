\documentclass[12pt, a4paper]{article}
\usepackage{titlesec}
\usepackage{lipsum}
\usepackage{amssymb}
\usepackage{amsmath}
\usepackage[margin=1.3in]{geometry}
\usepackage[mathscr]{euscript}
\usepackage{tabto}


\DeclareSymbolFont{rsfs}{U}{rsfs}{m}{n}
\DeclareSymbolFontAlphabet{\mathscrsfs}{rsfs}
\titlelabel{\thetitle.\quad}
\righthyphenmin=1000
\lefthyphenmin=1000

\begin{document}

\section*{Unit 1}
\vspace{1em}

\subsection*{LOS 1. Differentiate between linear and nonlinear PDEs}
Characteristic of linear PDE:
\begin{enumerate}
    \item $\mathcal{L}(u+v) = \mathcal{L}u + \mathcal{L}v$
    \item $\mathcal{L}(cu) = c\mathcal{L}u$
\end{enumerate}
Consequence of linearity: \\
Solution to homogeneous + solution to inhomogeneous = another solution to inhomogeneous\\
\vspace{0.3em}

\subsection*{LOS 2. Distinguish between ODEs and PDEs}
In general, ODE involves one dependent variable $y$ which is a function of $x$. On the other hand, PDE involves one dependent $u$ which is a function multiple independent variables $x, t, ...$ . \\ \\ 
Example of ODE:
\begin{gather*}
    \frac{\partial y}{\partial x} + y = 0
\end{gather*} \\
Example of PDE:
\begin{gather*}
    \frac{\partial u}{\partial x} + \frac{\partial u}{\partial t} = 0 \\
    u_{x} + u_{t} = 0
\end{gather*}
\vspace{0.3em}

\subsection*{LOS 3. Identify the degree of a given PDEs}
Degree of PDE is equal to the highest degree of derivative in the PDE.\\ \\
Degree 1: $u_{x} + u_{t} = 0$ \\
Degree 2: $u_{xx} + u_{t} = 0$ \\
Degree 3: $u_{xxx} + u_{t} = 0$ \\
... \\
\vspace{0.3em}


\subsection*{LOS 4. Solve first order PDE using geometric method for constant coefficients}
First order PDE with constant coefficients:
\begin{gather*}
    u_{x} + cu_{y} = 0
\end{gather*}
General Solution:
\begin{gather*}
    u(x, y) = f(cx - y)
\end{gather*}
\vspace{0.3em}

\subsection*{LOS 5. Solve first order PDE using geometric method for variable coefficients}
First order PDE with variable coefficients:
\begin{gather*}
    u_{x} + yu_{y} = 0
\end{gather*}
General Solution:
\begin{gather*}
    u(x, y) = \phi(ye^x)
\end{gather*}
Example:
\begin{gather*}
    u_{x} + yu_{y} = 0, u(x, 1) = x^2 \\
    u(x, y) = \phi(ye^{-x}) \\ 
    u(x, 1) = \phi(e^{-x}) = x^2 \\
    x = -\ln t \rightarrow \phi(t) = (-\ln t)^2 \\ 
    \therefore u(x, y) = (-\ln ye^{-x})^2 = (-\ln y + x)^2
\end{gather*}
\vspace{0.3em}

\subsection*{LOS 6. Distinguish between parabolic, hyperbolic and elliptic PDEs}
In this course, at maximum, we will have second order PDE. General formula of second order PDE:
\begin{gather*}
    a_{11}u_{xx} + 2a_{12}u_{xy} + a_{22}u_{yy} + a_{1}u_{x} + a_{2}u_{y} + a_{0}u = 0 \\
    a_{12} = a_{21} \\
    u_{xy} = u_{yx}
\end{gather*}
Type of PDE depends on determinant $\mathscrsfs{D} = a_{12}^2 - a_{11}a_{22}$:
\begin{enumerate}
    \item Elliptic: $\mathscrsfs{D} < 0$, all eigenvalues are positive
    \item Hyperbolic: $\mathscrsfs{D} > 0$, none of the eigenvalues vanish, one eigenvalue has the opposite sign from the $(n-1)$ others
    \item Parabolic: $\mathscrsfs{D} = 0$, one zero eigenvalues and the rest have the same signs
\end{enumerate}
\vspace{0.3em}

\subsection*{LOS 7. Identify different types of initial and boundary conditions}
Types of initial and boundary conditions:\\
\begin{itemize}
    \item (D) Dirichlet: $u(0) = \alpha, u(L) = \beta$
    \item (I) Initial: $u(0) = \alpha, u'(0) = \beta$
    \item (N) Neumann: $u_{x}(0) = \alpha, u_{x}(L) = \beta$
    \item (R) Robin: $\beta u_{x} + \alpha u(0) = 0$, linear combination of other initial or boundary conditions\\
\end{itemize}
\vspace{0.3em}

\subsection*{LOS 8. Analyze a PDE for being well-posed: existence, uniqueness and stability}
Well-posedness includes:
\begin{itemize}
    \item Existence: there exist a solution that can be expressed explicitly, solution provided by boundary conditions satisfy the PDE
    \item Uniqueness: the solution is not dependent on auxillary variables and the same across the defined range
    \item Stability: when data are changed very little, the corresponding solution also change very little
\end{itemize}
\vspace{1em}

\subsection*{LOS 9. Solve wave equation using the Fourier method and the operator method}
Wave equation:
\begin{gather*}
    u_{xx} - u_{tt} = 0
\end{gather*} \\
Fourier Method:
\begin{gather*}
    u_{\xi, \eta} = e^{i\xi t}e^{i\eta x} \qquad \xi, \eta \in \mathbb{R}    \\
    \mathcal{L} u_{\xi, \eta} = (-\xi^2 + \eta^2)u_{\xi, \eta} \\ 
    u(x, t) = \phi(x+t) - \psi(x-t) \\
    \phi(x) = \sum a_ke^{i\xi_k x} \qquad \psi(x) = \sum b_je^{i\xi_j x} \qquad a_k, b_j \in \mathbb{C} 
\end{gather*} \\
Operator Method:
\begin{itemize}
    \item Expand the PDE:
    \begin{gather*}
        \partial_{tt} - \partial_{xx} = (\partial_t - \partial_x)(\partial_t + \partial_x) \\ 
        y = (\partial_t - \partial_x)(\partial_t + \partial_x)
    \end{gather*} 
    \item Assume for solution $u(x, t)$ there is a function $v$ which satisfies the following:
    \begin{gather*}
        v \mid v(x, t) = (\partial_t + \partial_x) u(x, t) \\
        (\partial_t - \partial_x)v(x, t) = 0 \\ 
        v_x - v_t = 0\\
        v(x, t) = f(x + t)\\
        (\partial_t + \partial_x) u(x, t) = f(x + t)
    \end{gather*}
    \item By linearity:
    \begin{gather*}
        u = u^0 + w \\
        (\partial_t + \partial_x)u^0 = f(x + t) \qquad (\partial_t + \partial_x)w = 0
    \end{gather*}
    \item By method of characteristic line:
    \begin{gather*}
        (\partial_t + \partial_x)w = 0 \\
        w_t + w_x = 0 \\
        w(x, t) = g(x - t)
    \end{gather*}
    \item Substitute $z = x+t$ and assume $h' = f$:
    \begin{gather*}
        (\partial_t + \partial_x)u^0 = f(x + t) \\
        \frac{d}{dz}u^0(z) = f(z)\\
        u^0(z) = h(z)\\
        u^0(x+t) = \frac{h(x+t)}{2} \because (\partial_t + \partial_x)u^0 = \frac{h'(x+t)}{2} + \frac{h'(x+t)}{2} = f(x, t)
    \end{gather*}
    \item Substitute to the original equation:
    \begin{gather*}
        u(x, t) = \frac{h(x+t)}{2} + g(x-t)
    \end{gather*}
\end{itemize}
\vspace{0.3em}
    

\subsection*{LOS 10. Learn the principle of causality}
General solution to wave equation:
\begin{gather*}
    u_{xx} - u_{tt} = 0 \\
    u(x, 0) = \phi(x)\\
    u_t(x,0) = \psi(x)\\
    u(x, t) = f(x+t) + g(x-t) \qquad f, g \in C^2
\end{gather*} \\
Example with Dirichlet condition:
\begin{itemize}
    \item Problem statement:
    \begin{gather*}
        u_{xx} - u_{tt} = 0 \\
        u(x, 0) = \phi(x)\\
        u_t(x,0) = 0\\
    \end{gather*}
    \item Evaluate the boundary conditions:
    \begin{gather*}
        u(x, t) = f(x+t) + g(x-t) \\
        u(x, 0) = f(x) + g(x) = \phi(x)\qquad (1)\\
        u_t(x, 0) = f'(x+t) - g'(x-t) = f'(x) - g'(x) =0 \qquad (2)
    \end{gather*}
    \item Integrate $(2)$:
    \begin{gather*}
        f'(x) + g'(x) = 0 \\
        f(x) = g(x) + c
    \end{gather*}
    \item From $(1)$ and assume $c$ = 0:
    \begin{gather*}
        f(x) + g(x) = \phi(x) \\
        f(x) + f(x) - c = \phi(x)\\
        2f(x) = \phi(x) + c \\
        f(x) = \phi(x)/2 = g(x) \\
        \therefore u(x,t) = \frac{\phi(x+t) + \phi(x-t)}{2}
    \end{gather*}
    \item From $(1)$ and assume $c$ = 0:
    \begin{gather*}
        \phi(x) = u(x, 0) \\
        \therefore u(x,t) = \frac{u(x+t, 0) + u(x-t, 0)}{2}
    \end{gather*}
\end{itemize}
The above equation implies causality. The value of $u$ at point $t$ can be predicted if we know exactly the points at $t=0$, $u(x+t, 0)$ and $u(x-t, 0)$.

\vspace{0.3em}

\subsection*{LOS 11. Theorem on solution for wave equation}
Example with Neumann condition:
\begin{itemize}
    \item Problem statement:
    \begin{gather*}
        u_{xx} - u_{tt} = 0 \\
        u(x, 0) = 0\\
        u_t(x,0) = \psi(x)
    \end{gather*}
    \item Evaluate the boundary conditions:
    \begin{gather*}
        u(x, t) = f(x+t) + g(x-t) \\
        u(x, 0) = f(x) + g(x) = 0 \implies f = -g\\
        u_t(x, 0) = f'(x+t) - g'(x-t) \rightarrow 2f'(x) = \psi(x) \\
        f(x) = \frac{1}{2} \int_{-\infty}^x \psi(y)dy\\
        \therefore u(x,t) = \frac{1}{2} \left(\int_{-\infty}^{x+t} \psi(y)dy + \int_{-\infty}^{x-t} \psi(y)dy\right) = \frac{1}{2} \int_{[x+t, x-t]}\psi(y)dy
    \end{gather*}
\end{itemize}
Therefore, the general solution for the wave equation:
\begin{gather*}
    u_{xx} - u_{tt} = 0 \\
    u(x, 0) = \phi(x)\\
    u_t(x,0) = \psi(x)\\
    u(x, t) = \frac{\phi(x+t) + \phi(x-t)}{2} + \frac{1}{2} \int_{[x+t, x-t]}\psi(y)dy
\end{gather*} \\
Alternative formulation ($S^1$ is solution operator for Dirichlet condition and $S^2$ is solution operator for Neumann condition and $\dot{S}$ is the derivative):
\begin{gather*}
    S^1(\phi)(x, t) = \frac{\phi(x+t) + \phi(x-t)}{2}\\\
    S^1(\phi)(x, 0) = \phi(x) = Id\\
    \dot{S}^1(\phi)(x, 0) = 0
\end{gather*} 
\begin{gather*}
    S^2(\psi)(x, t) = \frac{1}{2} \int_{[x+t, x-t]}\psi(y)dy\\
    S^2(\psi)(x, 0) = 0\\
    \dot{S}^2(\psi)(x, 0) = \psi(x) = Id
\end{gather*}
\begin{gather*}
    \therefore S(\phi, \psi) = S^1(\phi) + S^2(\psi)
\end{gather*} \\

\vspace{0.3em}
\end{document}