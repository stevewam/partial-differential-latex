\documentclass[12pt, a4paper]{article}
\usepackage{titlesec}
\usepackage{lipsum}
\usepackage{amssymb}
\usepackage{amsmath}
\usepackage{bbm}
\usepackage[margin=1.3in]{geometry}
\usepackage[mathscr]{euscript}
\usepackage{tabto}


\DeclareSymbolFont{rsfs}{U}{rsfs}{m}{n}
\DeclareSymbolFontAlphabet{\mathscrsfs}{rsfs}
\titlelabel{\thetitle.\quad}
\righthyphenmin=1000
\lefthyphenmin=1000


\begin{document}
\section*{Unit 4}
\vspace{1em}

\subsection*{LOS 1. Determine the solution operator for a periodic differential equation}
\vspace{0.3em}
Problem:
\begin{gather*}
    u_t = \mathscrsfs{L}(u) \qquad -\pi<x<\pi, -\infty<t<\infty\\
    \mathscrsfs{L}(u)= \sum_{j=0}^{n} a_j \frac{\partial^j u}{\partial x^j}\\
    u(x, 0) = \varphi(x)\\
\end{gather*}
Use the following notation:
\begin{gather*}
    u^t(x) = u(x, t) \\
\end{gather*}
Solution:
\begin{gather*}
    u(x, t) = \sum_{k \in \mathbb{Z}} e^{tP(ik)}\hat{\varphi}(k)e^{ikx}
\end{gather*}
\vspace{0.3em}

\subsection*{LOS 2. Learn how to calculate eigen functions for a given PDE}
\subsection*{LOS 3. Solve a periodic differential equation using Fourier coefficients}
Eigenfunctions:
\begin{gather*}
    e_k(x) = e^{ikx}
\end{gather*}
\begin{enumerate}
    \item $\frac{d}{dx}e_k = (ik)e_k$
    \item $\left(\frac{d}{dx}\right)^j(e_k) = (ik)^je_k$
    \item $\mathscrsfs{L}(e_k) = P(ik)e_k$\\
\end{enumerate}
Application to PDE:
\begin{itemize}
    \item Problem:
    \begin{gather*}
        u_t = \mathscrsfs{L}(u) \\
        u^0 = \varphi
    \end{gather*}
    \item $\mathscrsfs{L}(\psi_k)$ operator the following orthonormal basis expansion (by rule, every function can be expanded as a generalized Fourier series):
    \begin{gather*}
        \mathscrsfs{L}(\psi_k) = \lambda_k\psi_k
    \end{gather*}
    \item Then, the general solution is:
    \begin{gather*}
        u(x, t) = \sum_{\mathbb{R}} e^{t\lambda_k}(\phi_k, \varphi)\psi_k
    \end{gather*}
\end{itemize}
Example:
\begin{itemize}
    \item Problem:
    \begin{gather*}
        u_t = \mathscrsfs{L}(u) \qquad -\pi<x<\pi, -\infty<t<\infty\\
    \mathscrsfs{L}(u)= \sum_{j=0}^{n} a_j \frac{\partial^j u}{\partial x^j}\\
    \end{gather*}
    \[ \varphi(x) = \begin{cases} 
        1 & -\frac{\pi}{2}<x< \frac{\pi}{2} \\
        0 & |x| > \frac{\pi}{2}
     \end{cases}\]
    \begin{gather*}
        \varphi(0) = \frac{1}{2}\\\\
        P(x) = x^2 \qquad \text{Polynomial for Diffusion}
    \end{gather*}
    \item Fourier dot product:
    \begin{gather*}
        \hat{\varphi}(k) = \int_{-\pi}^\pi e^{-ikx}\varphi(x) \frac{dx}{2\pi}
    \end{gather*}
    \item Solution:
    \begin{gather*}
        2\pi \hat{\varphi}(k) = \int_{-\pi}^\pi e^{-ikx}\varphi(x) dx\\
        2\pi \hat{\varphi}(k) = \int_{-\frac{\pi}{2}}^{\frac{\pi}{2}} e^{-ikx} dx \\
        2\pi \hat{\varphi}(k) = \left.\frac{e^{-ikx}}{-ik}\right\rvert_{-\frac{\pi}{2}}^{\frac{\pi}{2}} = \frac{e^{-ik\frac{\pi}{2}} - e^{ik\frac{\pi}{2}}}{-ik} = \frac{e^{ik\frac{\pi}{2}} - e^{-ik\frac{\pi}{2}}}{ik}
    \end{gather*}
    \item Identities:
    \begin{enumerate}
        \item For $k = 0$: $e^{ik\frac{\pi}{2}} = e^0 = 0$
        \item For $k = 1$: $e^{ik\frac{\pi}{2}} = e^{i\frac{\pi}{2}} = i$
        \item For $k = 2$: $e^{ik\frac{\pi}{2}} = e^{i\pi} = -1$
        \item For $k = -1$: $e^{ik\frac{\pi}{2}} = e^{-i\frac{\pi}{2}} = -i$
    \end{enumerate}
    \item Therefore:
    \[ 2\pi \hat{\varphi}(k)= \frac{e^{ik\frac{\pi}{2}} - e^{-ik\frac{\pi}{2}}}{ik} =  \begin{cases} 
        0 & k \text{ even} \\
        \frac{2}{k} & k = 4m+1\\
        -\frac{2}{k} & k =4m+3
    \end{cases}\]  
    \item Then, we plug this in into the original solution operator:
    \begin{gather*}
        u(x, t) = \sum_{k \in \mathbb{Z}} e^{tP(ik)}\hat{\varphi}(k)e^{ikx}
    \end{gather*}
\end{itemize}
\vspace{0.3em}

\subsection*{LOS 4. Solve a periodic differential equation with a source term using Fourier coefficients}
Problem:
\begin{gather*}
    u_t - \mathscrsfs{L}(u) = g \qquad -\pi<x<\pi, -\infty<t<\infty\\
    u^0 = \varphi(x)\\
    u(x, t) = u^t
\end{gather*}
Solution:
\begin{gather*}
    u^t = \int^t_0 S(t-s)g(s)ds + S(t)(\varphi)\\
    S(t)(f) = \sum_{k \in \mathbb{Z}}e^{tP(ik)}\hat{f}(k)e_k
\end{gather*}
Proof for $\varphi = 0$:
\begin{itemize}
    \item Solution ($x$ is not considered as it is being fixed at a single point):
    \begin{gather*}
        v(x, t) = v^t = \int_0^t S(t-s)g(s)ds
    \end{gather*}
    \item By Liebniz rule of integration:
    \begin{gather*}
        \frac{d}{dt}v = \int^t_0 \frac{d}{dt} S(t-s)g(s)ds + S(t-t)g(t) 
    \end{gather*}
    \item $S(t-t) = S(0)$ is equal to identity:
    \begin{gather*}
        \frac{d}{dt}v = \int^t_0 \frac{d}{dt} S(t-s)g(s)ds + g(t) 
    \end{gather*} 
    \item Evaluate $\int^t_0 \frac{d}{dt} S(t-s)g(s)ds$:
    \begin{gather*}
        \frac{d}{dt} S(t-s)g(s) = \frac{d}{dt}\sum_{k \in \mathbb{Z}}e^{tP(ik)}\hat{g}(k)e_k\\
        \frac{d}{dt} S(t-s)g(s)= \sum_{k \in \mathbb{Z}}e^{tP(ik)}\hat{g}(k)P(ik)e_k\\\\
        P(ik)e_k = \mathscrsfs{L}(e_k) \qquad \rightarrow\frac{d}{dt} S(t-s) = \sum_{k \in \mathbb{Z}}e^{tP(ik)}\hat{g}(k)\mathscrsfs{L}(e_k)\\
        \frac{d}{dt} S(t-s) = \mathscrsfs{L}\left(\sum_{k \in \mathbb{Z}}e^{tP(ik)}\hat{g}(k)e_k\right)\\
        \frac{d}{dt} S(t-s) = \mathscrsfs{L}S(t-s)
    \end{gather*}
    \item Therefore:
    \begin{gather*}
        \frac{d}{dt}v = \int^t_0 \mathscrsfs{L}S(t-s)g(s)ds + g(t)\\
        \frac{d}{dt}v = \mathscrsfs{L}\int^t_0 S(t-s)g(s)ds + g(t)\\
        \frac{d}{dt}v = \mathscrsfs{L}v + g(t)\\
    \end{gather*}
\end{itemize}
% Incomplete example:
% \begin{itemize}
%     \item Problem:
%     \begin{gather*}
%         P(x) = - x^4\\
%         u_t - \mathscrsfs{L}(u) = t\cos(x) \qquad -\pi<x<\pi, -\infty<t<\infty\\
%         u(x, 0)=\sin(x)
%     \end{gather*}
%     \item Using solution formula:
%     \begin{gather*}
%         u^t = \int_0^t S(t-s) s\cos(x) ds + S(t) \sin(x)
%     \end{gather*}
%     \item Identities:
%     \begin{gather*}
%         \cos(x) =\frac{e^{ix}+e^{-ix}}{2} \\
%         \sin(x) = \frac{e^{ix}-e^{-ix}}{2} 
%     \end{gather*}
%     \item Fourier coefficients:
%     \[\widehat{\cos}(k) = \begin{cases}
%         \frac{1}{2} & k = 1\\
%         \frac{1}{2} & k = -1\\
%         0 & \text{elsewhere}
%     \end{cases}\]\\
% \end{itemize}
General application:
\begin{itemize}
    \item Problem
    \begin{gather*}
        P_n(x) = a_0 + a_1x + a_2x^2 + ... + a_nx^n \\
        u_t = \mathscrsfs{L}(P_n(\frac{\partial}{\partial x}))(u) = a_0u + a_1u_x + ... + a_n u_{x...x} \qquad \text{x...x} \rightarrow \text{n times}
    \end{gather*}
    \item Solution operator:
    \begin{gather*}
        u(x, t) = S(t)(f) =\sum_{k \in \mathbb{Z}} e^{tP(ik)}\hat{f}(k)e^{ikx}
    \end{gather*}
    \item Case I:
    \begin{gather*}
        u_t = u_{xx} - iu_{xxx}\\
        P(x) = x^2 - ix^3\\
        e^{tP(ik)} = e^{-tk^2}e^{-tk^3} \rightarrow \text{Solution is stable as $e^{tP(ik)}$ vanishes as k increases}
    \end{gather*}
    \item Case II:
    \begin{gather*}
        u_t = u_{xx} + iu_{xxx}\\
        P(x) = x^2 + ix^3\\
        e^{tP(ik)} = e^{-tk^2}e^{tk^3} \rightarrow \text{Solution is not stable as $e^{tk^3}$ explodes as k increases}
    \end{gather*}
    \item Case III:
    \begin{gather*}
        u_t = u_{xx} \pm u_{xxx}\\
        P(x) = x^2 \pm x^3\\
        e^{tP(ik)} = e^{-tk^2}e^{\pm itk^3} \rightarrow \text{Solution is stable as $|e^{-tk^2}e^{\pm itk^3}| = e^{-2tk^2}$} \\
    \end{gather*}
\end{itemize}
Example:
\begin{itemize}
    \item Problem:
    \begin{gather*}
        u_t - u_{xxx} = t\cos(x) \qquad -\pi < x< \pi\\
        \mathscrsfs{L}(u) = u_{xxx}\\
        P(x) = x^3\\
        u(x, 0)=0
    \end{gather*}
    \item Solution operator:
    \begin{gather*}
        u^t = \int_0^t S(t-s)g(s)ds \qquad (1)\\
        u(x, t) = S(t)(f) =\sum_{k \in \mathbb{Z}} e^{tP(ik)}\hat{f}(k)e^{ikx}
    \end{gather*}
    \item Fourier coefficients:
    \begin{gather*}
        \varphi = \sum a_k e_k\\
        a_k = \hat{\varphi}(k) \\
        (e_j, f) = \sum_k a_k (e_j, e_k) = a_j
    \end{gather*}
    \item Therefore:
    \begin{gather*}
        \cos(x) =\frac{e^{ix}+e^{-ix}}{2}
    \end{gather*}
    \[\widehat{\cos}(k) = \begin{cases}
        \frac{1}{2} & k = 1\\
        \frac{1}{2} & k = -1\\
        0 & \text{elsewhere}
    \end{cases}\]\\
    \item Continuing on Equation (1):
    \begin{gather*}
        u^t = \int_0^t S(t-s)(s\cos(x))ds \\
        u^t = \int_0^t S(t-s)(\cos(x))(s)ds \\
        S(t-s)(\cos(x)) = \sum_{k \in \mathbb{Z}} e^{tP(ik)}\widehat{\cos}(k)e^{ikx} = \frac{1}{2}\left[e^{tP(i)}e_1 + e^{tP(-i)}e_{-1}\right]\\\\
        u^t = \int_0^t \frac{1}{2}\left[e^{(t-s)P(i)}e_1 + e^{(t-s)P(-i)}e_{-1}\right]sds\\
        u^t = \frac{1}{2}e_1\int_0^t e^{(t-s)P(i)}sds + \frac{1}{2}e_{-1}\int_0^t e^{(t-s)P(-i)}sds\\
    \end{gather*}
    \item Evaluate $\int_0^t e^{(t-s)P(i)}sds$ with integration by parts:
    \begin{gather*}
        \int_0^t e^{(t-s)P(i)}sds = e^{it}\left[\int_0^te^{-si}sds\right] = e^{it}\left[\left.\frac{e^{-si}s}{-i}\right\rvert^t_0 - \int_0^t\frac{e^{-si}s}{-i}ds\right]\\
        = e^{it}\left[-ie^{-ti}t - e^{-ti} + 1\right] = -it +e^{it} - 1
    \end{gather*}
    \item Therefore:
    \begin{gather*}
        u^t = \frac{1}{2}e_1(-it -e^{it} - 1) + \frac{1}{2}e_{-1}(it -e^{-it} + 1) \\
        u^t = it(\frac{e_1-e_{-1}}{2}) + (\frac{e_1-e_{-1}}{2}) + (\frac{e^{it}e_1 - e^{-it}e_{-1}}{2}) \\
        u^t = t\sin(x) + \cos(x) + i\sin(t+x)
    \end{gather*}
\end{itemize}
\vspace{0.3em}

\subsection*{LOS 5. Solve a diffusion equation using Fourier transformations}

Proof that a function can be its own Fourier coefficient:
\begin{itemize}
    \item Let:
    \begin{gather*}
        f(x) = e^{-\frac{x^2}{2}}\\
    \end{gather*}
    \item Suppose we want to form the following integral:
    \begin{gather*}
        \hat{f}(\xi) = \int_{\mathbb{R}}e^{ax}e^{-\frac{x^2}{2}}\frac{dx}{\sqrt{2\pi}} \qquad a \in \mathbb{R}\\
        \hat{f}(\xi) = \int_{\mathbb{R}}e^{ax-\frac{x^2}{2}}\frac{dx}{\sqrt{2\pi}} \\
        \frac{2ax-x^2}{2} = \frac{-(x^2-2ax+a^2-a^2)}{2} = \frac{-[(x-a)^2-a^2]}{2} = -\frac{(x-a)^2}{2}+\frac{a^2}{2}\\
        \therefore \hat{f}(\xi) = \int_{\mathbb{R}}e^{\frac{a^2}{2}}e^{-\frac{(x-a)^2}{2}}\frac{dx}{\sqrt{2\pi}}\\
        \hat{f}(\xi) = e^{\frac{a^2}{2}}\int_{\mathbb{R}}e^{-\frac{(x-a)^2}{2}}\frac{dx}{\sqrt{2\pi}}
    \end{gather*}
    \item Change of variable:
    \begin{gather*}
        \hat{f}(\xi) =\int_{\mathbb{R}}e^{-\frac{(x-a)^2}{2}}\frac{dx}{\sqrt{2\pi}}\\
        \hat{f}(\xi) =\int_{\mathbb{R}}e^{-\frac{y^2}{2}}\frac{dy}{\sqrt{2\pi}} = 1\\
        \hat{f}(\xi) = e^{\frac{a^2}{2}}\\\\
        a = i\xi\\
        \therefore \hat{f}(\xi) = e^{-\frac{\xi^2}{2}}
    \end{gather*}
\end{itemize}
Facts:
\begin{itemize}
    \item Parzeval equality:
    \begin{gather*}
        \|\hat{f}\|_{L_2} = \|f\|_{L_2} \Leftrightarrow \sum_{\mathbb{R}}|f(x)|^2\frac{dx}{2\pi}
    \end{gather*}
    \item Fourier coefficient for continuous transformation:
    \begin{gather*}
        \hat{f}(\xi) = \int_{\mathbb{R}}e^{-i\xi x}f(x) \frac{dx}{\sqrt{2\pi}}
    \end{gather*}
    \item Fourier inversion (recovery formula):
    \begin{gather*}
        {f}(x) = \int_{\mathbb{R}}e^{-i\xi x}\hat{f}(\xi) \frac{d \xi}{\sqrt{2\pi}}
    \end{gather*}
    \item Function is its own Fourier coefficient:
    \begin{enumerate}
        \item Gaussian function: $e^{-\frac{\xi^2}{2}} = \int_{\mathbb{R}}e^{-i\xi x}e^{-\frac{x^2}{4t}}\frac{dx}{\sqrt{2\pi}}$\\
        \item General: $e^{-t|\xi|^2} = \int_{\mathbb{R}^n}e^{-i\xi x}e^{-\frac{|x|^2}{2}}\frac{dx}{\sqrt{4\pi t}^n}$
    \end{enumerate}
\end{itemize}
General solution:
\begin{itemize}
    \item Problem:
    \begin{gather*}
        u_t = \mathscrsfs{L}_P(u)\\
        u(x, 0) = \varphi(x)
    \end{gather*}
    \item Fourier coefficient:
    \begin{gather*}
        \hat{u}^t(\xi) = \hat{u}(\xi, t) = \int_\mathbb{R}e^{-i\xi x}u(x, t)\frac{dx}{2\pi}
    \end{gather*}
    \item Observation 1 (Fourier transformation of $u$ applied by the differential operator is a polynomial times $\hat{u}$ as $u$ itself can be expanded using its orthogonal basis $\hat{u}$):
    \begin{gather*}
        \widehat{\mathscrsfs{L}_P(u)}(\xi) = P(i\xi)\hat{u}(\xi)\\
        u_t = P(i\xi)\hat{u}(\xi)\\
    \end{gather*}
    \item Observation 2:
    \begin{gather*}
        \widehat{\left(\frac{d}{dt}u\right)}(\xi) = (i\xi)\hat{u}(\xi)\\
    \end{gather*}
    \item Derivation:
    \begin{gather*}
        u_t = \mathscrsfs{L}_P(u)\\
        \widehat{\left(\frac{d}{dt}u\right)}(\xi) = \widehat{\mathscrsfs{L}_P(u)}(\xi)\\
        \frac{d}{dt}\hat{u}(\xi) = P(i\xi)\hat{u}(\xi) \qquad \rightarrow \text{ODE}\\
        \hat{u}(\xi, t) = e^{tP(i\xi)}\hat{u}(\xi, 0)\\
        \hat{u}(\xi, t) = e^{tP(i\xi)}\hat{\varphi}(\xi)\\
    \end{gather*}
    \item Using recovery formula:
    \begin{gather*}
        u(x, t) = \int e^{i\xi x}e^{tP(i\xi)}\hat{\varphi}(\xi)\frac{d\xi}{\sqrt{2\pi}}\\
    \end{gather*}
    \item Properties:
    \begin{enumerate}
        \item Solution works in $L_2$ space (the above is an $L_2$ solution)
        \item For a nice $\varphi$ (infinitely differetiable and decaying at $\infty$), solution is unique\\
    \end{enumerate}
\end{itemize}
Solution for heat equation:
\begin{itemize}
    \item Problem:
    \begin{gather*}
        u_t = u_{xx}\\
        u(x, 0) = \varphi(x)
    \end{gather*}
    \item Solution:
    \begin{gather*}
        u(x, t) = \int_{\mathbb{R}} e^{i\xi x}e^{tP(i\xi)}\hat{\varphi}(\xi)\frac{d\xi}{\sqrt{2\pi}}\\
        u(x, t) = \int_{\mathbb{R}} e^{i\xi x}e^{-t\xi^2}\hat{\varphi}(\xi)\frac{d\xi}{\sqrt{2\pi}}\\
        u(x, t) = \int_{\mathbb{R}} e^{i\xi x}\left(\int_{\mathbb{R}}e^{-i\xi y}e^{-\frac{y^2}{2}}\frac{dy}{\sqrt{4\pi t}}\right)\hat{\varphi}(\xi)\frac{d\xi}{\sqrt{2\pi}}\\
        u(x, t) = \int_{\mathbb{R}} \left(\int_{\mathbb{R}} e^{i\xi(x-y)}\hat{\varphi}(\xi)\frac{d\xi}{\sqrt{2\pi}}\right)e^{-\frac{y^2}{2}}\frac{dy}{\sqrt{4\pi t}}\\
        u(x, t) = \int_{\mathbb{R}} \varphi(x-y)e^{-\frac{y^2}{2}}\frac{dy}{\sqrt{4\pi t}} \qquad \Rightarrow S(t)(\varphi)(x)
    \end{gather*}
    \item Alternative solution expression (by change of variable):
    \begin{gather*}
        u(x, t) = \int_{\mathbb{R}} \varphi(z)e^{-\frac{(x-z)^2}{4t}}\frac{dz}{\sqrt{4\pi t}} \qquad \Rightarrow S(t)(\varphi)(x)
    \end{gather*}

\end{itemize}
% Application:
% \begin{gather*}
%     u_t = u_{xxx} \qquad \rightarrow\text{on the upper half plane $t > 0$}\\
%     u^t(x)= \int_{\mathbb{R}}e^{-i\xi x}\hat{u}^t(\xi) d\xi\\
%     \hat{u}^t(\xi, t) = \int_{\mathbb{R}}e^{-i\xi x}u^t(x, t) \frac{dx}{\sqrt{2\pi}}\\\\
%     e_\xi(x) = e^{i\xi x}\\
%     e_{xxx} = (i\xi)^2e_\xi = -i\xi^3e_\xi\\\\
%     \frac{d}{dt}\hat{u}^t(\xi) = P(\xi)\hat{u}^t(\xi)\\
%     u^t(\xi) = e^{tP(\xi)}u^t(\xi)
% \end{gather*}
% Example for simple polynomial:
% \begin{gather*}
%     P(x) = x\\
%     \widehat{\left(\frac{d}{dx}u\right)}(\xi) = \int_{\mathbb{R}} e^{-i\xi x}u'(x)\frac{dx}{\sqrt{2\pi}}\ \qquad \rightarrow \text{fix $t=0$}\\\\
\tiny $\text{Integration by parts $\int_\mathbb{R} fg' dx = \left.fg\right\rvert_\mathbb{R} - \int_\mathbb{R} f'gdx$}$
%     \widehat{\left(\frac{d}{dx}u\right)}(\xi) = -\int_{\mathbb{R}} (-i\xi)e^{-i\xi x}u(x)\frac{dx}{\sqrt{2\pi}}\ \\
%     \widehat{\left(\frac{d}{dx}u\right)}(\xi) = (i\xi)\int_{\mathbb{R}} e^{-i\xi x}u(x)\frac{dx}{\sqrt{2\pi}}\\
%     \widehat{\left(\frac{d}{dx}u\right)}(\xi) = (i\xi)\hat{u}(\xi)
% \end{gather*}
\vspace{0.3em}
\end{document}