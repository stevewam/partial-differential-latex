\documentclass[12pt, a4paper]{article}
\usepackage{titlesec}
\usepackage{lipsum}
\usepackage{amssymb}
\usepackage{amsmath}
\usepackage{bbm}
\usepackage[margin=1.3in]{geometry}
\usepackage[mathscr]{euscript}
\usepackage{tabto}


\DeclareSymbolFont{rsfs}{U}{rsfs}{m}{n}
\DeclareSymbolFontAlphabet{\mathscrsfs}{rsfs}
\titlelabel{\thetitle.\quad}
\righthyphenmin=1000
\lefthyphenmin=1000


\begin{document}
\section*{Unit 5}
\vspace{1em}

\subsection*{LOS 1. Compare the solutions for continuous case vs periodic case for a given PDE}
Continuous case:
\begin{itemize}
    \item Problem: 
    \begin{gather*}
        u_t = u_{xxxxxx} \qquad -\infty < x < \infty\\
        u(x, 0) = \phi(x) = \int_{\mathbb{R}} e^{i\xi x}\hat{\phi}(k)\\
        P(x) = x^6\\
    \end{gather*}
    \item Assume $t$ is fixed:
    \begin{gather*}
        \frac{d}{dt} \hat{u}(\xi, t) =(i\xi)^6\hat{u}({\xi, t}) \\
        \hat{u}(\xi, t) = e^{-t\xi^6}\hat{u}(\xi, 0) \\
    \end{gather*}
    \item By Fourier Inversion formula:
    \begin{gather*}
        u(x, t) = \int_{\mathbb{R}} e^{i\xi x}\hat{u}(\xi, t)\frac{d\xi}{\sqrt{2}\pi}\\
        u(x, t) = \int_{\mathbb{R}} e^{i\xi x}e^{t\xi^6}\hat{u}(\xi, 0)\frac{d\xi}{\sqrt{2}\pi} \qquad (1)\\
    \end{gather*}
    \item By Fourier expansion:
    \begin{gather*}
        f(\xi) = e^{\xi^6} = \int_{\mathbb{R}} e^{i\xi z}\hat{f}(z)\frac{dz}{\sqrt{2}\pi}\\
        e^{\xi^6} = \int_{\mathbb{R}} e^{-i\xi z}\hat{f}(z)\frac{dz}{\sqrt{2}\pi} \qquad \text{since $\hat{f}$ is even}\\
        e^{t\xi^6} = e^{(t^{\frac{1}{6}}\xi)^6} = \int_{\mathbb{R}} e^{-it^{\frac{1}{6}}\xi z}\hat{f}(z)\frac{dz}{\sqrt{2}\pi}\\
    \end{gather*}
    \item By change of variable:
    \begin{gather*}
        y = t^{\frac{1}{6}}z\\
        e^{t\xi^6} = \int_{\mathbb{R}} e^{-i\xi y}\hat{f}\left(\frac{y}{t^\frac{1}{6}}\right)\frac{dy}{t^\frac{1}{6}\sqrt{2}\pi}\\
        h_t(y) = \frac{1}{t^\frac{1}{6}\sqrt{2}\pi}\hat{f}\left(\frac{y}{t^\frac{1}{6}}\right)\\
        \therefore e^{t\xi^6} = \int_{\mathbb{R}} e^{-i\xi y}h_t(y)dy\\
    \end{gather*}
    \item Substitute back to Equation (1):
    \begin{gather*}
        u(x, t) = \int_{\mathbb{R}} e^{i\xi x}e^{t\xi^6}\hat{u}(\xi, 0)\frac{d\xi}{\sqrt{2}\pi} \\ 
        u(x, t) = \int_{\mathbb{R}} e^{i\xi x}\int_{\mathbb{R}} e^{-i\xi y}h_t(y)dy\;\hat{u}(\xi, 0)\frac{d\xi}{\sqrt{2}\pi}\\
        u(x, t) = \int_{\mathbb{R}} \left[\int_{\mathbb{R}} e^{i\xi (x-y)}\hat{u}(\xi, 0)\frac{d\xi}{\sqrt{2}\pi}\right]h_t(y)dy\\
        u(x, t) = \int_{\mathbb{R}} u(x-y, 0) h_t(y)dy\\
        u(x, t) = \int_{\mathbb{R}} \phi(x-y) h_t(y)dy\\
    \end{gather*}
\end{itemize}
Periodic case:
\begin{itemize}
    \item Problem:
    \begin{gather*}
        u_t = u_{xxxxxx} \qquad -\pi < x < \pi\\
        u(x, 0) = \phi(x) = \sum_{k \in \mathbb{Z}} e^{ikx}\hat{\phi}(\xi)\\
        P(x) = x^6\\
    \end{gather*}
    \item Assume $t$ is fixed:
    \begin{gather*}
        \frac{d}{dt} \hat{u}(k, t) =(ik)^6\hat{u}{k, t} \\
        \hat{u}(k, t) = e^{-tk^6}\hat{u}(k, 0) \\
    \end{gather*}
    \item By Fourier Inversion formula:
    \begin{gather*}
        u(x, t) = \sum_{k \in \mathbb{Z}} e^{-tk^6}\hat{u}(k, 0)e^{ikx} \qquad (1)\\
    \end{gather*}
    \item Let:
    \begin{gather*}
        H_t(x) = \frac{1}{2\pi}\sum_{k \in  \mathbb{Z}} e^{-tk^6}e^{ikx}\\
        e^{-tk^6} = \hat{H_t(x)} = \int_{-\pi}^{\pi}e^{-iky}H_t(y)dy
    \end{gather*}
    \item Substitute back to Equation (1):
    \begin{gather*}
        u(x, t) = \sum_{k \in \mathbb{Z}} \left[\int_{-\pi}^{\pi}e^{-iky}H_t(y)dy\right]\hat{u}(k, 0)e^{ikx}\\ 
        u(x, t) = \int_{-\pi}^{\pi}\left[\sum_{k \in \mathbb{Z}} e^{ik(x-y)}\hat{u}(k, 0)\right]H_t(y)dy\\
        u(x, t) = \int_{-\pi}^{\pi}u(x-y, 0)H_t(y)dy\\
        u(x, t) = \int_{-\pi}^{\pi}\phi(x-y)H_t(y)dy\\
    \end{gather*}
\end{itemize}
In general:
\begin{itemize}
    \item Upper half plane (continuous):
    \begin{gather*}
        u(x, t) = \int_{\mathbb{R}} \phi_{ext}(x-y) h_t(y)dy = \int_{\mathbb{R}} \phi_{ext}(y) h_t(x-y)dy\\
    \end{gather*}
    \item Periodic:
    \begin{gather*}
        u(x, t) = \int_{-\pi}^{\pi}\phi(x-y)H_t(y)dy = \int_{-\pi}^{\pi}\phi(y)H_t(x-y)dy\\
    \end{gather*}
\end{itemize}
Example:
\begin{itemize}
    \item Problem:
    \begin{gather*}
        u_t = ku_{xx} \\
        H_t(y) = \frac{c}{\sqrt{kt}}\sum_{j\in\mathbb{Z}}e^{-\frac{|y-j|^2}{4kt}}\qquad -\pi < x < \pi \\
        h_t(y) = \frac{1}{\sqrt{4\pi kt}}e^{-\frac{y^2}{4kt}}\qquad -\infty < x < \infty 
    \end{gather*}
\end{itemize}
\vspace{0.3em}

\subsection*{LOS 2. Analyze the continuity of a solution of a PDE}
Solution formula:
\begin{gather*}
    u_t = ku_{xx} \qquad -\infty < x < \infty\\
    u(x, t) = \int_{\mathbb{R}} \phi(x-y) \frac{1}{\sqrt{4\pi kt}}e^{-\frac{y^2}{4kt}}dy = \int_{\mathbb{R}} \phi(y) \frac{1}{\sqrt{4\pi kt}}e^{-\frac{(x-y)^2}{4kt}}dy\\
\end{gather*}
If $\phi$ is continuous and bounded:
\begin{gather*}
    \lim_{t \to 0} u(x, t) = \phi(x)\\
    \sup_z |\phi(z)| \leq K
\end{gather*}
Proof:
\begin{itemize}
    \item Let $x \in \mathbb{R}$ since $\phi$ is continuous at $x$ $\exists \delta > 0$ $\forall y$ where $|y-x| < \delta \Rightarrow |\phi(x)-\phi(y)|<\epsilon$ where $\epsilon > 0$
    \item Evaluate the difference:
    \begin{gather*}
        \int_{\mathbb{R}} \frac{1}{\sqrt{4\pi kt}}e^{-\frac{y^2}{4kt}}dy = 1\\
        u(x, t) - \phi(x) = \int_{\mathbb{R}} \phi(x-y) \frac{1}{\sqrt{4\pi kt}}e^{-\frac{y^2}{4kt}}dy - \int_{\mathbb{R}} \phi(x) \frac{1}{\sqrt{4\pi kt}}e^{-\frac{y^2}{4kt}}dy\\
        u(x, t) - \phi(x)= \int_{\mathbb{R}} [\phi(x-y) - \phi(x)] \frac{1}{\sqrt{4\pi kt}}e^{-\frac{y^2}{4kt}}dy\\
        u(x, t) - \phi(x)= \int_{|y| < \delta} [\phi(x-y) - \phi(x)] \frac{1}{\sqrt{4\pi kt}}e^{-\frac{y^2}{4kt}}dy \\
        \qquad \qquad \qquad\qquad \qquad + \int_{|y| > \delta} [\phi(x-y) - \phi(x)] \frac{1}{\sqrt{4\pi kt}}e^{-\frac{y^2}{4kt}}dy\\
        |u(x, t) - \phi(x)|\leq \int_{|y| < \delta} [\epsilon] \frac{1}{\sqrt{4\pi kt}}e^{-\frac{y^2}{4kt}}dy \\
        \qquad \qquad \qquad\qquad \qquad + \int_{|y| > \delta} [K + K] \frac{1}{\sqrt{4\pi kt}}e^{-\frac{y^2}{4kt}}dy\\
    \end{gather*}
    \item Change of variable:
    \begin{gather*}
        z^2 = \frac{y^2}{4t} \Rightarrow z = \frac{y}{\sqrt{2t}} \Rightarrow dz = \frac{dy}{\sqrt{2t}}\\
        |u(x, t) - \phi(x)|\leq \epsilon\int_{|y| < \delta} \frac{1}{\sqrt{4\pi kt}}e^{-\frac{y^2}{4kt}}dy + 2K \int_{|z| > \frac{\delta}{\sqrt{2t}}} e^{-\frac{z^2}{2}}\frac{dy}{\sqrt{2\pi}}\\
        |u(x, t) - \phi(x)|\leq \epsilon+ 2Ke^{-\frac{\delta}{t}}\\
        |u(x, t) - \phi(x)|\leq \epsilon+ \epsilon \qquad \text{$2Ke^{-\frac{\delta}{t}} \to \epsilon$ when $t$ is large}\\
        \therefore |u(x, t) - \phi(x)|\leq 2\epsilon
    \end{gather*}
\end{itemize}
\vspace{0.3em}

\subsection*{LOS 3. Describe the maximum principle for heat equation}
Theorem:
\begin{gather*}
    u_t = ku_{xx} \qquad 0<x<l\\
    \sup_{(x, t) \in \hat{\Omega}} u(x, t) = \sup_{(x, t) \in \delta \Omega} u(x, t) \qquad \text{where $\delta\Omega$ is the boundary enclosing $\hat{\Omega}$}\\
\end{gather*}
Proof:
\begin{itemize}
    \item Introduce a perturbation term where $\epsilon>0$:
    \begin{gather*}
        v(x, t) = u(x, t) + \epsilon x^2 \\
    \end{gather*}
    \item If $(x_0, t_0)$ is an interior point:
    \begin{gather*}
        v_t(x_0, t_0) = 0 \qquad \text{First derivative at maximum is 0}\\
        v_xx(x_0, t_0) \leq 0 \qquad \text{Second derivative at maximum is negative or 0}\\
        \therefore \underbrace{v_t - v_{xx}}_{\geq 0} = \underbrace{u_t - u_{xx}}_{=0} - \underbrace{2k\epsilon}_{>0}\\
        \underbrace{v_t - v_{xx}}_{\geq 0} = \underbrace{u_t - u_{xx} - 2k\epsilon}_{<0}\\
    \end{gather*}
    \item LHS and RHS contradicts. Therefore, maximum for $v$ cannot be an interior point (1).
    \item Prove for $u$:
    \begin{gather*}
        \sup_{(x, t) \in \hat{\Omega}} u(x, t) \leq \sup_{(x, t) \in \hat{\Omega}} v(x, t) \qquad \text{Since $\epsilon > 0$}\\ 
        \sup_{(x, t) \in \hat{\Omega}} u(x, t) \leq \sup_{(x, t) \in \delta \Omega} v(x, t) \qquad \text{By proof in (1)}\\ 
        \sup_{(x, t) \in \hat{\Omega}} u(x, t) \leq \sup_{(x, t) \in \delta \Omega} u(x, t) + \sup_{(x, t) \in \delta \Omega} \epsilon x^2\\
        \sup_{(x, t) \in \hat{\Omega}} u(x, t) \leq \sup_{(x, t) \in \delta \Omega} u(x, t) + \epsilon l^2\\
        \sup_{(x, t) \in \hat{\Omega}} u(x, t) \leq \sup_{(x, t) \in \delta \Omega} u(x, t) \qquad \text{When $\epsilon \to 0$}\\
    \end{gather*}
    \item Maximum principle: maximum exists in the boundary.\\
\end{itemize}
Application:
\begin{itemize}
    \item Theorem: Heat equation has a unique solution
    \begin{gather*}
        u_t - ku_{xx} = f \qquad k > 0\\
        u(x, 0) = \phi(x)\\
        u(0, t) = h(t)\\
        u(l, t) = k(t)
    \end{gather*}
    \item Proof: Assume $u^1$ and $u^2$ are solution to the heat equation:
    \begin{gather*}
        w = u^1 - u^2\\
        w_t = kw_{xx}\\
        w(x, 0) = \phi^1(x) - \phi^2(x) = 0\\
        w(0, t) = h^1(x) - h^2(x) = 0\\
        w(l, t) = k^1(x) - k^2(x) = 0\\
        \max_{(x, t)} w(x, t) \leq 0 \qquad \text{By maximum principle}\\
        \max_{(x, t)} -w(x, t) \leq 0 \qquad \text{By maximum principle}\\
        \therefore w \equiv 0 \Rightarrow \text{$u^1$ and $u^2$ are equal}
    \end{gather*}
    
\end{itemize}
\vspace{0.3em}

\end{document}